\documentclass[12pt]{article}
\usepackage[margin = 2.4cm]{geometry} % For margins of 3cm
\usepackage{float} % For H float position
\usepackage{gensymb} % For some symbols
\usepackage{amsfonts, amssymb, amsmath} % All three for maths symbols
\usepackage[export]{adjustbox} % For figure frames
\setlength{\parskip}{6pt} % To make nice looking paragraph spacing
\usepackage[export]{adjustbox} % For figure frames
\usepackage[numbers, super, sort&compress]{natbib} % bibliographies
\usepackage{setspace} % Allows double spacing

\doublespacing % Makes all lines (not figure legends) double spaced

\title{Methylation status of B cells in Transient Hypogammaglobulinemia of Infancy}
\date{Research Hypothesis and Experimental Proposal}
\author{Brittany Howell \\ a1646948}

\newcommand{\CDH}{CD4$^+$ } %Thelper
\newcommand{\CDR}{CD4$^+$CD25$^+$ } %Treg
\newcommand{\IFNG}{IFN-$\gamma$ }
\newcommand{\NFKB}{NF-$\kappa$B }
\newcommand{\naive}{na\"{i}ve }

\newcommand{\Th}{T$_\textrm{H}$ }
\newcommand{\ThO}{T$_\textrm{H}1$} 
\newcommand{\ThT}{T$_\textrm{H}2$ }
\newcommand{\Tc}{T$_\textrm{C}$ }

\begin{document}
	\maketitle
	
	\paragraph{Hypothesis}
	~\\
	B cells from Transient Hypogammaglobulinemia of Infancy (THI) patients will exhibit more DNA methylation, indicating incomplete B cell differentiation, than age-matched controls.
	
	\section{Background}
	
		Immunoglobulins are a vital component of the adaptive immune system \citep{Simon15}.  
		The B cells which produce antibodies (active immunoglobulin) in adults are not fully mature in young infants, resulting in a decrease in serum immunoglobulin levels after birth \citep{Martin10,Rechavi15,Stiemh80}. 
		Physiologic hypogammaglobulinemia refers to the point when serum immunoglobulin reaches its lowest point, commonly at 4-6 months of age \citep{Dressler89}. 
		THI is a disorder whereby affected persons have a prolongation or exacerbation of regular hypogammaglobulinemia, followed by spontaneous recovery \citep{Stiemh80,Dressler89,AlHerz14,Gitlin56,AlHerz11,Rosen66,McGeady87, Dalal98,Tiller78,Buckley83}.
		The mechanism causing low serum immunoglobulin in THI patients has not yet been elucidated \citep{AlHerz14}. 

	\newpage 
	
		\paragraph{Cause of THI} 
			~\\
			Immunoglobulin deficiency can result from B cell precursors failing to either mature into B cells or further fail to differentiate into antibody secreting plasma cells \citep{Fiorilli86}. 
			Studies investigating THI have found that levels of circulating B cells are normal and subpopulations of B cells are intact \citep{Tiller78,Stiemh80,Siegel81,Buckley83,Fiorilli86,Dressler89}.
			With no obvious B cell deficiency, the cause of THI has been speculated extensively, but no proposed cause has been supported by replicated evidence \citep{Fudenberg64,Rosen66,Nathenson71,Willenbockel60,Soothill68,Tiller78,Fiorilli86,Ovadia14,Siegel81,McGeady87}.
			
			In regards to genetic inheritance, THI was initially thought to be familial \citep{Willenbockel60}. 
			\citet{Soothill68} proposed that THI was a manifestation of genetic heterozygosity for other immunodeficiency diseases, noting the high number of patients who had immunodeficient relatives.
			While it remains a possibility as noted by \citet{McGeady87}, no proceeding studies have shown supporting evidence \citep{Tiller78,Fiorilli86, Ovadia14}.
			
			
	\section{Lineage commitment}

		Activation and differentiation of T cells is governed greatly by epigenetic changes which insure the phenotype of the T cell \citep{Zeng13}.
		DNA methylation was the first epigenetic mechanism recognised, and the one that is most extensively studied \citep{Begin14}. 
		In T regulatory cells (Treg), the methylation status of the Treg-specific demethylated region (TSDR) is imperative in Treg differentiation \citep{Polansky08}.
		In the thymus, where T cells mature, Tregs are induced by T cell receptor engagement. 
		Subsequent demethylation occurs at the TSDR allowing FOXP3 to bind to its own gene to stabilise FOXP3 expression, stabilising differentiation to Treg.
		FOXP3 is also expressed during the activation of other T cell subsets, but due to the methylation of the TSDR, FOXP3 expression is transient \citep{Ohkura13}.
		Therefore, demethylation permits FOXP3 binding and thence confirms Treg lineage.
		
	\section{Proposal}
	
		The most intriguing feature of THI is its self-limited nature; recurrent infections gradually subside and serum IgG levels increase with no obvious cause  \citep{Tiller78,Soothill68,Siegel81,McGeady87,Dressler89,Kowalczyk97,Dalal98}. 
		Furthermore, the lack of evidence supporting a genetic basis suggests that the cause of THI is not within the genome \citep{Tiller78,Fiorilli86,Ovadia14}
		
		In common variable immunodeficiency, a disease related to THI, some B cells resemble immature B cells, producing very little IgG \citep{Fiorilli86}. 
		Incomplete maturation caused lack of IgG production, so it is possible that the delayed onset of IgG synthesis in THI is also due to incomplete B cell maturation.
		Activation and differentiation is greatly influenced by epigenetic changes; latent maturity could be caused by inappropriate methylation of B cell development or differentiation genes. 

		To study incomplete lineage commitment, B cells will be sampled from THI patients and age-matched controls and characterised using whole-genome bisulfite sequencing.
		As in \citet{Kulis15}, DNA methylation maps will be generated for sorted human B cell populations: uncommitted haematopoietic progenitor cells (HPCs), pre-BII cells, \naive B cells from peripheral blood, germinal center B cells, memory B cells from peripheral blood and plasma cells from bone marrow.
		Global demethylation normally occurs as B cells mature \citep{Oakes16}. 
		If methylation is a cause of delayed maturation, the methylome of THI patients should be distinct to the age-matched controls.
		
		
	\bibliographystyle{BrittSuperScript}
	\bibliography{../literature}
	
\end{document}