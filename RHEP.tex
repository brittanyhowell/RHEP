\documentclass[12pt]{article}
\usepackage[margin = 2.4cm]{geometry} % For margins of 3cm
\usepackage{float} % For H float position
\usepackage{gensymb} % For some symbols
\usepackage{amsfonts, amssymb, amsmath} % All three for maths symbols
\usepackage[export]{adjustbox} % For figure frames
\setlength{\parskip}{6pt} % To make nice looking paragraph spacing
\usepackage[export]{adjustbox} % For figure frames
\usepackage[numbers, super, sort&compress]{natbib} % bibliographies
\usepackage{setspace} % Allows double spacing
\usepackage{pdfpages} % Allows including PDFs
\usepackage{longtable} % Allows longtables
\usepackage{lscape} % To allow some pages to be in landscape

% Longtable settings
\setlength\LTleft{0pt} % flush left
\setlength\LTright{0pt} % flush right
\setlength{\LTcapwidth}{8in}


\doublespacing % Makes all lines (not figure legends) double spaced

\title{Methylation status of B cells in Transient Hypogammaglobulinemia of Infancy}
\date{}
\author{Brittany Howell}

\newcommand{\CDH}{CD4$^+$ } %Thelper
\newcommand{\CDR}{CD4$^+$CD25$^+$ } %Treg
\newcommand{\IFNG}{IFN-$\gamma$ }
\newcommand{\NFKB}{NF-$\kappa$B }
\newcommand{\naive}{na\"{i}ve }

\newcommand{\Th}{T$_\textrm{H}$ }
\newcommand{\ThO}{T$_\textrm{H}1$} 
\newcommand{\ThT}{T$_\textrm{H}2$ }
\newcommand{\Tc}{T$_\textrm{C}$ }

%%%%%%%%%%%%%%%%%%%%%%%%%%%%%%%%%%%%%%%
%%%%%%%%%  Word count: 2500  %%%%%%%%%%  
%%%%%  Inc. all legends & titles  %%%%%  
%%%%%%%%%%%%%%%%%%%%%%%%%%%%%%%%%%%%%%%


\begin{document}
	\maketitle
	
	\paragraph{Hypothesis}
	~\\
	B cells from Transient Hypogammaglobulinemia of Infancy (THI) patients exhibit more DNA methylation, indicating incomplete B cell differentiation, than individuals without THI.
	
	\section{Background}
	
		Immunoglobulins are a vital component of the adaptive immune system \citep{Simon15}.  
		The B cells which produce antibodies (active immunoglobulin) in adults are not fully mature in young infants, hence serum immunoglobulin levels decrease after birth \citep{Martin10,Rechavi15,Stiemh80}. 
		Physiologic hypogammaglobulinemia refers to the point when serum immunoglobulin reaches its lowest level, commonly at 4 to 6 months of age \citep{Dressler89}. 
		THI is a disorder where regular hypogammaglobulinemia is prolonged or exacerbated then spontaneously alleviated. \citep{Stiemh80,Dressler89,AlHerz14,Gitlin56,AlHerz11,Rosen66,McGeady87, Dalal98,Tiller78,Buckley83}.
		The mechanism causing low serum immunoglobulin in THI patients is unknown \citep{AlHerz14}. 

	
		\paragraph{Cause of THI} 
			~\\
			Immunoglobulin deficiency can result from B cell precursors failing to develop into mature B cells or mature B cells failing to differentiate into antibody secreting plasma cells \citep{Fiorilli86}. 
			Studies investigating THI have found that levels of circulating B cells are normal and subpopulations of B cells are intact \citep{Tiller78,Stiemh80,Siegel81,Buckley83,Fiorilli86,Dressler89}.
			With no obvious B cell deficiency, the cause of THI has been speculated extensively, but no proposed cause has been supported by replicated evidence \citep{Fudenberg64,Rosen66,Nathenson71,Willenbockel60,Soothill68,Tiller78,Fiorilli86,Ovadia14,Siegel81,McGeady87}.
			
			In regards to genetic inheritance, THI was initially thought to be familial \citep{Willenbockel60}. 
			\citet{Soothill68} proposed that THI was a manifestation of genetic heterozygosity for other immunodeficiency diseases, noting the high number of patients who had immunodeficient relatives.
			While it remains a possibility, no proceeding studies have shown supporting evidence \citep{Tiller78,Fiorilli86, Ovadia14}.
			
			
		\subsection{Lineage commitment}
	
			Activation and differentiation of T cells is governed greatly by epigenetic changes which affirm the lineage \citep{Zeng13}.
			DNA methylation was both the first epigenetic mechanism recognised, and the one most extensively studied \citep{Begin14}. 
			In T regulatory cells (Treg), the methylation status of the Treg-specific demethylated region (TSDR) is imperative in Treg differentiation \citep{Polansky08}.
			Tregs are induced by T cell receptor engagement which causes demethylation at the TSDR.
			FOXP3, the Treg inducing protein, can then bind to its own gene stabilising its expression and hence stabilising differentiation to Treg lineage.
			FOXP3 is also expressed during the activation of other T cell subsets, but due to the methylation of the TSDR, FOXP3 expression is transient \citep{Ohkura13}.
			Therefore, demethylation permits FOXP3 binding and thence confirms Treg lineage.
	
%	\section{Proposal}
%
%		
%			The most intriguing feature of THI is its self-limited nature: recurrent infections gradually subside and serum IgG levels increase with no obvious cause  \citep{Tiller78,Soothill68,Siegel81,McGeady87,Dressler89,Kowalczyk97,Dalal98}. 
%			Furthermore, the lack of evidence supporting a genetic basis suggests that the cause of THI is not within the genome \citep{Tiller78,Fiorilli86,Ovadia14}.
%			
%		
%			In common variable immunodeficiency, a disease related to THI, some B cells resemble immature B cells producing very little IgG \citep{Fiorilli86}. 
%			Incomplete maturation results in limited IgG production, so it is possible that the delayed onset of IgG synthesis in THI is also due to incomplete B cell maturation.
%			Development and differentiation are greatly influenced by epigenetic changes; latent maturity could be caused by inappropriate methylation of B cell development or differentiation genes. 
%	
%			To study incomplete lineage commitment B cells will be sampled from THI patients and age-matched controls then characterised using whole-genome bisulfite sequencing.
%			As in \citet{Kulis15} DNA methylation maps will be generated for sorted human B cell populations: uncommitted haematopoietic progenitor cells, pre-BII cells, plasma cells from bone marrow, germinal center B cells, \naive B cells from peripheral blood and memory B cells from peripheral blood.
%			Global demethylation normally occurs as B cells mature \citep{Oakes16}. 
%			If methylation is a cause of delayed maturation, the methylome of THI patients should be distinct to the age-matched controls.
			
	\section{Experimental aims}
		
		\paragraph{Experimental aim 1:} Using whole genome bisulfite sequencing, determine the regions of the genome which show differences between THI patients and normal individuals. (Include somewhere that I'd like to look at the THI patients over time)
		
		\paragraph{Experimental aim 2:} Using ChIP-Seq, determine if the transcription factor families (INSERT FAMILIES HERE) are differentially methylated between THI patients and normal individuals (find some TFs involved in IgG production)
		
		\paragraph{Experimental aim 3:} Using RNA-Seq, determine if (INSERT FAMILY) are differentially expressed between THI patients and normal individuals. (find some genes involved in IgG production)

	\section{Data collection and management}
	
		\subsection{Collecting samples}
		
			\begin{itemize}
				\item \citet{Ovadia14} will provide samples of tissue from 26 THI patients (Collected between 1976 and 1994). Samples were collected by the group from various hospitals, immunology clinics, and past PID trials. Information about the samples is patchy (see table \ref{table:samples}) 
				\item Further THI samples will be collected from the WCH Immunology clinic
				\item A full clinical record will be kept for every sample, they will be monitored for as long as they give consent between initial onset of presentation and normalisation of IgG levels
				\item given that occurrence is 1 per 10$^6$ births, we can expect to collect X samples over 5 years. More would be possible with the participation of other hospitals around the country. 
			\end{itemize}
		
		\subsection{Matching controls} 
			
			\begin{itemize}
				\item The immune system is dynamic, it is important that, as much as possible, we can find disease related differences rather than developmental changes
				\item For this reason, the individuals will be matched according to as many criteria as possible. 
				\item Age will be the most important factor, secondly breastfeeding/bottle feeding, then things like diet, gender, ethnicity and that kind of thing. 
				\item When collecting the control samples, all individuals will be kept until at least 1 year, then they will vary depending on their matched sample
				\item In the best case, every THI patient will have a matched control for comparison with the same ethnicity, background, diet and breastfeeding status etc. However depending on the samples we will make concessions to allow analysis. Age will hopefully never have to be messed with.
			\end{itemize}
		
		\subsection{How B cells will be segregated into development stages}
			
			\begin{itemize}
				\item Hey look FACS
				\item Should refer to someone's cell sorting
				\item Should include a figure of the different stages
				\item Use \citeauthor{Kulis15}
			\end{itemize}
		
	
	\section{Aim 1:}
		
		\paragraph{Hypothesis:} DNA of B cells from THI patients will have differentially methylated regions compared to normal individuals.
		
		\subsection{Proposed experiment}
			
			\begin{itemize}
				\item Conduct whole genome bisulphite sequencing, 
				\item Refer to \citeauthor{Kulis15} as much as possible
			\end{itemize}
		
		\subsection{Possible outcomes and interpretations}		
			
			\begin{itemize}
				\item Methylome maps will be produced for every sample. 
				\item Hence, we will be able to compare across two vectors: normal vs THI and normal as time progresses
				\item I plan to produce a normal vs THI map for each age, and see the differences ebb as the IgG levels normalise
			\end{itemize}
	
	\section{Aim 2:}
	
		\paragraph{Hypothesis}
		
		\subsection{Proposed experiment}
		
		\subsection{Possible outcomes and interpretations}		
		
	\section{Aim 3:}
	
		\subsection{Hypothesis}
		
		\subsection{Proposed experiment}
		
		\subsection{Possible outcomes and interpretations}				
	

\appendix
	\section{Appendices}
		
		\subsection{Sample information}

			
			\begin{landscape} % Sample table
				\footnotesize
				\begin{longtable}[c]{|l | l |l|l|l|l|l|l|}
					\caption{Clinical details of patients with THI. Abbreviations: m, months; y, years; CVI, common variable immunodeficiency; -, unknown; def, d.} \\ \hline 	
					Individual & Tissue & Age at test  & Gender  & Ethnicity   & Breastfeeding status & Mode of birth & Family history of PID    \\ \hline \hline
					\endhead
					\hline 	\endfoot 
					\label{table:samples}
					1  & Whole Blood & 7m, 9m, 11m, 15m             & Female & Caucasian & Breastfed  & Vaginal           & None                 \\
					1  & Tonsils     & 7m                           &        &           &            &                   &  \\
					2  & Whole Blood & 8m, 11m, 1y 1m, 1y 3m, 1y 6m & Male   & Caucasian & Breastfed  & Vaginal           & None                 \\
					2  & Tonsils     & 7m                           & Male   & Caucasian & Breastfed  & Vaginal           & Sister with CVI      \\
					3  & Whole Blood & 9m, 11m, 1y 3m               & Male   & Jewish    & Bottle fed & Caesarian section & None                 \\
					3  & Tonsils     & 1y2m                         &        &           &            &                   &  \\
					4  & Whole Blood & 10m, 1y 1m,  1y 8m           & Male   & Caucasian & Bottle fed & Vaginal           & None                 \\
					4  & Tonsils     & 1y 11m                       & Male   & Caucasian & Breastfed  & Vaginal           & Brother with CVI     \\
					5  & Whole Blood & 11m, 1y 3m, 1y 5m, 1y 8m     & Male   & Asian     & Breastfed  & Vaginal           & Brother with CVI     \\
					6  & Whole Blood & 7m, 9m, 11m, 1y 8m           & Female & Jewish    & Breastfed  & Vaginal           & None                 \\
					7  & Whole Blood & 9m, 11m, 1y 2m, 1y 6m        & Male   & -         & Breastfed  & -                 & -                    \\
					8  & Whole Blood & 7m, 9m,                      & Female & -         & Breastfed  & -                 & -                    \\
					9  & Whole Blood & 7m, 9m, 11m                  & Male   & -         & Breastfed  & -                 & -                    \\
					10 & Whole Blood & 11m, 1y 2m, 1y 6m            & Male   & -         & Breastfed  & -                 & -                    \\
					11 & Whole Blood & 11m, 1y 1m, 1y 3m 1y 8m      & Male   & -         & Bottle fed & -                 & -                    \\
					12 & Whole Blood & 7m, 9m, 11m,                 & Female & Jewish    & Breastfed  & Vaginal           & None                 \\
					13 & Whole Blood & 9m, 11m, 1y 2m, 1y 6m        & Male   & Jewish    & Bottle fed & Vaginal           & None                 \\
					14 & Whole Blood & 7m, 9m, 1y 1m                & Female & Caucasian & Bottle fed & Vaginal           & None                 \\
					15 & Whole Blood & 7m, 11m, 1y 1m               & Male   & Caucasian & Breastfed  &                   & Sister with IgA def. \\
					16 & Whole Blood & 8m, 1y 1m, 1y 4m, 1y 6m      & Male   & -         & -          & Vaginal           & -                    \\
					17 & Whole Blood & 9m, 1y 2m, 1y 6m             & Male   & -         & -          & Vaginal           & -                    \\
					18 & Whole Blood & 7m, 9m, 1y 1m                & Female & -         & -          & Caesarian section & -                    \\
					19 & Whole Blood & 7m, 9m, 11m                  & Male   & -         & -          & Vaginal           & -                    \\
					20 & Whole Blood & 11m, 1y 2m, 1y 6m            & Male   & -         & -          & Vaginal           & -                    \\
					21 & Whole Blood & 11m, 1y 3m, 1y 5m, 1y 8m     & Male   & -         & -          & Caesarian section & -                    \\
					22 & Whole Blood & 7m, 9m, 11m, 1y 8m           & Female & -         & -          & -                 & None                 \\
					23 & Whole Blood & 9m, 1y 2m, 1y 6m             & Male   & -         & -          & -                 & None                 \\
					24 & Whole Blood & 7m, 9m, 1y 1m                & Female & -         & -          & -                 & None                 \\
					25 & Whole Blood & 7m, 11m,                     & Male   & -         & -          & -                 & Sister with IgA def. \\
					26 & Whole Blood & 8m, 1y 1m,                   & Male   & -         & -          & -                 & None                 \\
				\end{longtable}
			\end{landscape}	
				
	\bibliographystyle{BrittSuperScript}
	\bibliography{../literature}
	
	\includepdf[pages=-]{../Reading/Transient_hypogammaglobulinemia_of_infancy_Ovadia_and_Dalal_2014.pdf}
	\includepdf[pages=-]{../Reading/Whole-genome-fingerprint-of-the–DNA-methylome-during-human-B-cell-differentiation.pdf}
	
	
	
\end{document}