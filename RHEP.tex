\documentclass[12pt]{article}
\usepackage[margin = 2.4cm]{geometry} % For margins of 3cm
\usepackage{gensymb} % For some symbols
\usepackage{amsfonts, amssymb, amsmath} % All three for maths symbols
\usepackage[export]{adjustbox} % For figure frames
\setlength{\parskip}{6pt} % To make nice looking paragraph spacing
\usepackage[export]{adjustbox} % For figure frames
\usepackage[numbers, super, sort&compress]{natbib} % bibliographies
\usepackage{setspace} % Allows double spacing
\usepackage{pdfpages} % Allows including PDFs
\usepackage{longtable} % Allows longtables
\usepackage{lscape} % To allow some pages to be in landscape
\usepackage{float, subfloat} % For H flag and subfloats
\usepackage{caption, subcaption} % for ABCD Subcaptions
\usepackage[framemethod=tikz]{mdframed} % Allows coloured boxes
\hyphenpenalty 100000

% Longtable settings
\setlength\LTleft{0pt} % flush left
\setlength\LTright{0pt} % flush right
\setlength{\LTcapwidth}{8in}

% mdframed settings
\newmdenv[innerlinewidth=1.5pt, roundcorner=4pt,linecolor=mycolor,innerleftmargin=20pt,
innerrightmargin=20pt,innertopmargin=20pt,innerbottommargin=20pt,leftmargin =30, rightmargin=30]{mybox}
\definecolor{mycolor}{rgb}{0.122, 0.435, 0.698}


\doublespacing % Makes all lines (not figure legends) double spaced

\title{\vspace{-2cm} Transcriptome and epigenome profile of B cells in patients with Transient Hypogammaglobulinemia of Infancy}
\date{}
\author{Brittany Howell}

\newcommand{\CDH}{CD4$^+$ } %Thelper
\newcommand{\CDR}{CD4$^+$CD25$^+$ } %Treg
\newcommand{\IFNG}{IFN-$\gamma$ }
\newcommand{\NFKB}{NF-$\kappa$B }
\newcommand{\naive}{na\"{i}ve }
\newcommand{\Naive}{Na\"{i}ve }

\newcommand{\Th}{T$_\textrm{H}$ }
\newcommand{\ThO}{T$_\textrm{H}1$} 
\newcommand{\ThT}{T$_\textrm{H}2$ }
\newcommand{\Tc}{T$_\textrm{C}$ }

%%%%%%%%%%%%%%%%%%%%%%%%%%%%%%%%%%%%%%%
%%%%%%%%%  Word count: 2500  %%%%%%%%%%  
%%%%%  Inc. all legends & titles  %%%%%  
%%%%%%%%%%%%%%%%%%%%%%%%%%%%%%%%%%%%%%%


\begin{document}
	\maketitle
	
	\paragraph{Hypothesis}
	~\\
	Patients with Transient Hypogammaglobulinemia of Infancy (THI) will exhibit delayed loss of methylation in B cell lymphopoiesis genes, resulting in a deficiency of mature B cell subpopulations.
	
	\section{Background}
		
		Antibodies are a vital component of the adaptive immune system. 
		The production of antibodies occurs when \naive B cells are activated in response to foreign antigens \citep{Simon15}.
		After birth the maternal antibodies are degraded before infant B cells can mature, leading to a decline in serum antibody levels \citep{Hasselquist09,Martin10,Rechavi15}.
		Physiological hypogammaglobulinemia refers to the point when serum antibody concentration reaches its lowest level, commonly at four to six months of age \citep{Dressler89}.
		THI is a disorder where regular hypogammaglobulinemia is prolonged or exacerbated then spontaneously alleviated. \citep{Dressler89,AlHerz14,Gitlin56,AlHerz11,Rosen66,McGeady87,Stiemh80, Dalal98,Tiller78,Buckley83}.
		The mechanism causing THI is at present unknown \citep{AlHerz14}. 

			
		\paragraph{Cause of THI} 
			~\\
			Studies investigating THI have not identified any deficiencies in the antibody production pathway.
			Levels of circulating B cells have been reported as normal with subpopulations of B cells also intact \citep{Tiller78,Stiemh80,Siegel81,Buckley83,Fiorilli86,Dressler89}.
			Furthermore, upon antigenic challenge, most patients produce a normal antibody response \citep{Tiller78,Stiemh80,Buckley83,McGeady87,Dressler89,Dalal98}.
			THI was initially thought to be hereditary \citep{Willenbockel60, Soothill68}, however following studies have not shown supporting evidence \citep{Tiller78,Fiorilli86,Ovadia14}.
			With apparently normal B cell populations and no genetic basis, the cause of THI has been extensively speculated, but no proposed mechanism has been supported by replicated evidence \citep{Fudenberg64,Rosen66,Nathenson71,Willenbockel60,Soothill68,Tiller78,Fiorilli86,Ovadia14,Siegel81,McGeady87}.
			The most agreed upon cause is a delay in B cell maturation or activation \citep{McGeady87,Stiemh80,Walker94,Rosen84}. 
			
			B lymphocytes develop in the bone marrow from hematopoietic precursors \citep{Cooper15}.
			Development and maturation progress through stages labelled in figure \ref{fig:BCellDevelopment}. 
			Plasma cells produce the antibodies required for humoral immunity.
			Failure to proceed at any stage of B cell development can result in a deficiency of plasma cells and thence antibody deficiency.
			
			
			\begin{figure}[tb]
				\centering
				\includegraphics[width=\linewidth]{../Figures/Bcelldevelopment.png}
				\caption{B cell development from hematopoietic stem cell to memory and plasma cells. Phases are shown as antigen independent or dependent and location (bone marrow or periphery) is indicated. IGM/G/A/E indicates option of antibody isotype. BCR: B cell receptor.}
				\label{fig:BCellDevelopment}
			\end{figure}
			
			B lymphocyte differentiation is tightly regulated by transcription factors (TFs).
			Figure \ref{fig:TFBcell} shows the complex network of auto-regulation, cross regulation, and positive and negative feedback loops \citep{Choukrallah14,Polansky08,Oakes16,Zan15,Li13,Mercer11,Kulis15,McManus11}.
			TFs prominent in early stages of B cell specification and commitment include Pax5, E2A, Ebf1, Ikaros, PU.1, and FoxO1 \citep{Gao09,Maier04,Walter08,Decker09,Lin10,McManus11,Treiber10,Zandi00}. 
			Loss of function studies involving these TFs show some defect in B cell lineage commitment, resulting in a loss of a B cell subset or in some cases the entire B cell lineage \citep{Choukrallah14}.

			
			\begin{figure}
				\centering
				\includegraphics[width = 0.4\linewidth]{../Figures/BcellTF.png}
				\caption{Schematic representation of interactions between early B cell transcription factors. Adapted from \citet{Choukrallah14}. Arrows indicate positive regulation, direct positive regulation is indicated by reciprocal arrows.}
				\label{fig:TFBcell}
			\end{figure}
		
		
			\paragraph{Flow Cytometry}
			~\\
			Previous investigations into THI claimed that subpopulations of B cells are the same in THI patients as controls \citep{Tiller78,Stiemh80,Siegel81,Buckley83,Fiorilli86,Dressler89}. 
			The claim is supported by studies which distinguished only between mature and immature B cells. 
			Analysis techniques have improved substantially since the most recent experimental THI study took place.
			In particular flow cytometry (FACS) has developed extensively, increasing the number of measured parameters per cell \citep{Saeys16}.
			18-parameter FACS is now routinely used \citep{Perfetto04}, 30-parameter flow cytometers are becoming commercially available, and 50-parameter FACS is predicted to be available soon \citep{Saeys16}.
			An increased number of measurable parameters allows more comprehensive study of B cell subpopulations.
			
			
		\paragraph{Lineage commitment}
		~\\
			Epigenetic modifications act in concert with TFs to confer the phenotype of many cell subsets of the immune system \citep{Lara14,Zan15,Mercer11}. 
			A prominent example is the activation and differentiation of the many T cell subsets \citep{Begin14, Zeng13}. 
			Confirmation of the T regulatory cell (Treg) lineage relies on stable expression of the TF FOXP3.
			However, FOXP3 expression occurs in all T cell subsets upon activation \citep{Ohkura13,Polansky08}.
			The distinguishing factor between Tregs and other subsets is the methylation status of the FOXP3 gene, otherwise known as the Treg-specific demethylated region (TSDR) \cite{Shimazu16}.
			When the T cell receptor is engaged in Treg progenitors, demethylation occurs at the TSDR. 
			FOXP3 can then bind to its own gene stabilising expression, reinforcing commitment to the Treg lineage.
			Contrastingly, in other subsets the TSDR remains methylated and so FOXP3 expression is transient \citep{Ohkura13}.
			Hence methylation status of key genes can permit lineage commitment. 
			
			Aberrant epigenetics have recently been implicated as the cause of Common Variable Immunodeficiency (CVID), a disease similar to THI \citep{Tallmadge15}.
			CVID is a late-onset primary immunodeficiency characterised by dysfunction or loss of B lymphocytes and decreased immunoglobulin production \citep{Cunningham-Rundles12}.
			% R NA-Seq analysis identified 103 genes which were differentially expressed between healthy controls and CVID patients \citep{Tallmadge15}.
			Epigenome and transcriptome analysis revealed that the Pax5 enhancer was hypermethylated and severely  down-regulated.
%			The most severely down-regulated gene was the transcription factor Pax5.
			Pax5 is essential to commit a cell to the B cell identity through activation of 170 B cell specific genes and repression of at least 110 lineage inappropriate genes \citep{Schebesta07,Delogu06,Roessler07}.
			\citet{Tallmadge15} proposed that the cause of CVID was the methylation of Pax5. 
			If the methylation of the Pax5 enhancer was silencing the gene as proposed, a decline of B lymphopoiesis would occur in the bone marrow, followed by a depletion of B cells characteristic of CVID.
			
			It is possible that the prolonged antibody deficiency exhibited in THI patients is caused by a similar mechanism of epigenetic silencing.
			Epigenetics encompasses an extensive list of mechanisms including small RNA regulation, DNA methylation, chromatin modifications \citep{Zan15,Bodak14,Eichten14,Li13}.
			DNA methylation was the first epigenetic mechanism recognised, and is the most commonly studied \citep{Begin14}.
			Hence studying methylation of THI samples is a logical first step to understanding THI.
		
	\section{Summary and experimental aims}

		\begin{mybox}
			
			\begin{itemize}

				\item THI is a self-limited disorder characterised by prolonged deficient levels of serum antibody which spontaneously subside \citep{Tiller78,Soothill68,Siegel81,McGeady87,Dressler89,Kowalczyk97,Dalal98,AlHerz14,Gitlin56,AlHerz11,Rosen66,Stiemh80,Buckley83}.
				\item The claim of intact B cell subpopulations in THI is supported by outdated methods which only distinguish mature from immature B cells, providing good reasoning for a more comprehensive study.
				\item \citet{Tallmadge15} provided evidence that in CVID, aberrant methylation of key B cell maturation region, Pax5, was the cause of late onset dysfunctional B lymphopoiesis.
				
			\end{itemize}

		\end{mybox}
		
			\paragraph{Aim 1:} Using FACS, describe the B cell subpopulations in THI patients and normal individuals throughout early development.
			
			\paragraph{Aim 2:} Using whole genome bisulphite sequencing, identify regions of the genome which are differentially methylated in THI samples and controls.
			
			\paragraph{Aim 3:} Using RNA-Seq, identify regions which are differentially expressed in THI samples and controls.

	\section{Data collection and management}
	
		\subsection{Collecting samples}
			
			Whole blood and tonsil samples from 26 THI patients has been donated from the Wolfson Medical centre Pediatric Department (table \ref{table:samples}).
			Further peripheral blood, bone marrow and tonsil samples will be obtained from the Women's and Children's Hospital Immunodeficiency clinic. 
			Blood will be taken from THI patients between initial presentation and normalisation of antibody levels.
			Bone marrow will be taken in six monthly intervals following presentation until normalisation of antibody levels.
			Control samples will come from donations by healthy infants. 
			
		\subsection{Matching controls} 
		
			To determine B cell maturation changes caused by THI, it is imperative to compare samples at the closest possible developmental stages.
%			Maturation of B cells integrates numerous internal and environmental signals, so controlling for developmental stage has significant challenges. 
			Where possible, THI samples will be fully matched to control samples. 
			The most important criteria to match will be age, as it is the most prominent indicator of immune system development. 
			The method of both birth (vaginal or caesarian) and feeding (bottle or breastfed) will be considered as both have a large influence over the immune system \citep{Jakobsson14,Cho13,Brandtzaeg03,Rogier14,Gomez14}.
			Factors such as gender and ethnicity will also be matched if possible. 
			Diets will be standardised in all new patients.
	
			In the samples provided, listed in table \ref{table:samples}, there are gaps in the clinical history of samples.
			Without information such as ethnicity or mode of birth, it is impossible to match each sample to an appropriate control. 
			Additionally, the control samples will be taken from participants' donations which may not match the clinical history of any THI patient.
	%		There are a myriad of further external factors that will affect the dynamic nature of the immune system which are not controlled for, so even with stringent matching it is difficult to ensure samples are perfectly matched.
			While completely matching samples and controls is preferable, there are limitations in the scope of the matching. 
	
	\section{Aim 1: {\large Using FACS, describe the B cell subpopulations in THI patients and normal individuals throughout early development}}
	
		\paragraph{Hypothesis:} THI samples will be deficient in mature B cell subsets.
		
		\subsection{Proposed experiment}
		
				To analyse the distribution of B cell subpopulations, we will segregate the B cells using flow cytometry as in \citet{Kulis15} and \citet{Oakes16}.
				B cell populations obtained will include progenitor, pre-BI, pre-BII, immature and plasma cells from bone marrow, \naive and memory B cells from peripheral blood and plasma cells and germinal centre and \naive B cells from tonsils. 
				FACS will be used for all three tissues to sort populations using the surface markers in figure \ref{fig:BCellSorting}.
				
				\begin{figure}[tb]
					\centering
					\includegraphics[width=.9\linewidth]{../Figures/BcellSorting}
					\caption{Description of the FACS sorting markers used to segregate B cell populations.}
					\label{fig:BCellSorting} 
				\end{figure}

		
		\subsection{Possible outcomes and interpretations}		
		
		
			Segregating samples into B cell subpopulations serves two purposes. 
			Firstly, sorting means we can compare cells at the same maturity stage.
			It is imperative that any changes observed between THI and controls are caused by THI rather than normal B cell maturation changes.
			%After sorting, we can compare THI and control cells that at the same maturation stage. 
			%Therefore the developmental variation between compared cells will be reduced allowing more confidence that observed changes are due to THI.
			While there are always limitations when applying discrete stages to continuous processes, segregating B cells into such specific subtypes significantly reduces the chance of detecting differences which are due to normal development. 
			
			Secondly, as the cells will be separated based on maturity, we are able to analyse the distribution of subtypes. 
			If there is a deficiency of mature B cells in THI samples relative to controls, it could indicate a block in the B cell development pathway, as in \citet{Tallmadge15}.
			Alternatively, the distribution of B cell subpopulations could be intact, as claimed previously \citep{Tiller78,Stiemh80,Siegel81,Buckley83,Fiorilli86,Dressler89}.
			If each subset is intact it would indicate that the cause of THI may not be an error in B cell development, and perhaps investigation into antibody production or degradation should follow. 
			Therefore FACS allows us to compare cells more accurately and can show if the mechanism affects B cell development.
			
	\newpage
	\section{Aim 2: {\large Using whole genome bisulphite sequencing, identify regions of the genome which are differentially methylated in THI samples and controls}}
	
		\paragraph{Hypothesis:} B cells from THI patients will exhibit hypermethylation at key B cell development regions.
		
		\subsection{Proposed experiment}
		
			To investigate global DNA methylation of THI and control samples, we will produce full methylomes of the B cell lineages in figure \ref{fig:BCellSorting}. 
			Whole genome bisulfite sequencing (WGBS) will be used to obtain base-pair resolution of all methylated cytosines within the genome \citep{Kulis15,Oakes16}. 
			To perform the analysis, we will use two sets of biological replicates for each of the samples. 
			Samples will undergo two rounds of bisulfite conversion to ensure a cytosine to thymine conversion rate of over 99\%.
			Treated samples will be sequenced on an Illumina HiSeq 2000 platform and mapped to the genome using the STAR algorithm (v2.4.2a) \citep{Dobin13}.
			
			We will then use ChIP-seq data from the ENCODE project \citep{ENCODE-Project-Consortium12} to analyse methylation status in the context of transcription factor binding sites.
			The relative enrichment of each TFBS in any differentially methylated regions will be calculated in comparison to background reads. 
			A Fisher's exact test will be used to assign an odds ratio and \textit{P} value to each comparison. 
			Of particular focus will be the genes in figure \ref{fig:TFBcell} which are specific to B lymphopoiesis.

	\begin{figure}[tb]
		\centering
			%					\flushleft
			\includegraphics[width=.55\linewidth]{../Figures/methylomeMap}
			\caption{Methylome map produced by \citet{Kulis15}. }
			\label{fig:MethMap}

		\caption{Cytosine methylation status is shown for six stages of B cell development. Concentric circles are labelled with developmental stage. Colour corresponds to methylation status indicated in the scale.}
	\end{figure}
			
		
		\subsection{Possible outcomes and interpretations}

			Unbiased DNA methylation maps as in figure \ref{fig:MethMap} will be produced for each of the sorted cell populations.
			It is expected that in both THI and control samples in all age groups, global methylation decreases as the B cells mature \citep{Kulis15,Lai13,Kulis12,Shaknovich11}. 
			If THI causes delayed loss of methylation, it would be evident in methylation maps.
			By comparing the same individuals at different ages, we expect that regions of hypermethylation will gradually deplete and resemble the controls.
			
			The ChIP-Seq data provides a more detailed view of key regions such as transcription factor binding sites (TFBSs).
			We will produce a heatmap which displays the correlations between TFBSs and differentially methylated regions. 
			In the THI samples, we expect to find that regions related to B cell maturation or antibody production will be hypermethylated compared to background levels. 
			The location of differentially methylated regions (DMR) will indicate any potential causes of THI.
			If a region associated with B cell maturation, such as Pax5, is found to be differentially methylated it indicates that THI could be a result of a developmental block like that seen in CVID \citep{Tallmadge15}.
			Alternatively if there is a DMR associated with gene rearrangement, THI could be caused by a lack of B cell diversity.
			Any DMRs identified can be further investigated with expression analysis.
			
		
	\section{Aim 3: {\large Using RNA-Seq, identify regions which are differentially expressed in THI samples and controls}}
	
		\paragraph{Hypothesis:}
			
			Samples from THI patients will exhibit decreased expression for B cell commitment associated genes compared to controls.
					
		\subsection{Proposed experiment}
			
			To investigate expression changes across the genome of THI samples, we will conduct RNA-Seq analysis on each of the B cell subsets in figure \ref{fig:BCellSorting}. 
			RNA-Seq libraries will be generated using the TruSeq Stranded Total RNA kit (Illumina). 
			Sequenced reads will be aligned to the genome using the STAR algorithm (v2.4.2a)\citep{Dobin13}.
			qPCR analysis of gene expression will then be undertaken to quantify reads. 
			Target gene expression will be presented relative to average expression for the housekeeping genes \textit{GAPDH}, \textit{ACTB} and \textit{HPRT1}.
			
		\subsection{Possible outcomes and interpretations}				
			
			Expression analysis will allow us to  further investigate any DMRs found in methylation analysis as well as find other genes which are differentially expressed. 
			Heatmaps will be generated to display the genes which change expression between THI and control samples.
			
			We expect to see differentially expressed regions associated with B cell development or antibody production. 
			For example, if Rag1/2 was found to have decreased expression, it could indicate that impaired gene rearrangement is causing THI, as gene rearrangement is vital for antibody diversity \citep{Choukrallah14}.
			
			Genes which show hypermethylation in WGBS analysis are expected to have decreased expression in THI samples than controls. 
			If expression changes occur in regions that are not differentially methylated, the implicated region could possibly be affected by a different epigenetic mechanism such as a chromatin modification.
			Any expression changes not correlated with methylation status could be further investigated as potential causes of THI.
			

			
	\section{Conclusion}
	
		It is feasible that THI is caused by a delay in maturation as a result of aberrant methylation. 
		FACS, WGBS and RNA-Seq provide a broad survey of B lymphopoiesis in THI.
		If THI is caused by stunted maturation, our data will show where it occurs and will indicate any affected pathways.
		Should the results show that methylation is not involved in THI, the FACS and transcriptome data provide avenues for further research.


	\section{Appendices}
	\appendix

	
	\section{Sample information}

					\begin{landscape} % Sample table
						\footnotesize
						\begin{longtable}[c]{|l | l |p{4cm}|l|l|l|l|l|}
							\caption{Clinical details of patients with THI. Abbreviations: m, months; y, years; CVI, common variable immunodeficiency; -, unknown; def, d.} \\ \hline 	
							Individual & Tissue & Age at test  & Gender  & Ethnicity   & Breastfeeding status & Mode of birth & Family history of PID    \\ \hline \hline
							\endfirsthead
							\hline
							Individual & Tissue & Age at test  & Gender  & Ethnicity   & Breastfeeding status & Mode of birth & Family history of PID    \\ \hline \hline
							\endhead
							\hline 	\endfoot 
							\label{table:samples}
							1  & Whole Blood & 7m, 11m, 1y 7m, 2y 1m                                    & Female & Caucasian & Breastfed  & Vaginal           & None                 \\
							& Tonsils     & 2y 1m                                                    &        &           &            &                   &  \\ \hline
							2  & Whole Blood & 8m, 11m, 1y 3m, 1y 6m, 1y 8m, 2y 1m, 2y 4m, 2y 6m, 2y 8m & Male   & Caucasian & Breastfed  & Vaginal           & None                 \\
							& Tonsils     & 2y 6m                                                    &        &           &            &                   &  \\ \hline
							3  & Whole Blood & 9m, 1y 3m, 1y 9m, 2y 6m, 2y 9m, 3y 2m, 3y 6m             & Male   & Jewish    & Bottle fed & Caesarian section & None                 \\
							& Tonsils     & 3y 2m                                                    &        &           &            &                   &  \\ \hline
							4  & Whole Blood & 10m, 1y 1m, 1y 8m                                        & Male   & Caucasian & Bottle fed & Vaginal           & None                 \\ \hline
							5  & Whole Blood & 9m, 11m, 1y 3m, 1y 5m, 1y 8m                             & Male   & Asian     & Breastfed  & Vaginal           & Brother with CVI     \\ \hline
							6  & Whole Blood & 7m, 9m, 11m, 1y 8m                                       & Female & Jewish    & Breastfed  & Vaginal           & None                 \\ \hline
							7  & Whole Blood & 9m, 11m, 1y 2m, 1y 6m                                    & Male   & -         & Breastfed  & -                 & -                    \\ \hline
							8  & Whole Blood & 7m, 9m,                                                  & Female & -         & Breastfed  & -                 & -                    \\ \hline
							9  & Whole Blood & 7m, 9m, 11m, 1y 3m 1y 8m                                 & Male   & -         & Breastfed  & -                 & -                    \\ \hline
							10 & Whole Blood & 7m, 9m, 11m, 1y 2m, 1y 6m                                & Male   & -         & Breastfed  & -                 & -                    \\ \hline
							11 & Whole Blood & 9m, 11m, 1y 1m, 1y 3m 1y 8m                              & Male   & -         & Bottle fed & -                 & -                    \\ \hline
							12 & Whole Blood & 7m, 11m, 1y 2m, 1y 11m, 2y 6m, 2y 9m, 3y 4m, 3y 7m, 4y   & Female & Jewish    & Breastfed  & Vaginal           & None                 \\
							& Tonsils     & 2y 11m                                                   &        &           &            &                   &  \\ \hline
							13 & Whole Blood & 9m, 11m, 1y 2m, 1y 6m, 2y 4m, 2y 8m                      & Male   & Jewish    & Bottle fed & Vaginal           & None                 \\ \hline
							14 & Whole Blood & 9m, 1y 1m, 1y 7m, 2y 2m, 2y 9m, 3y 2m                    & Female & Caucasian & Bottle fed & Vaginal           & None                 \\
							& Tonsils     & 3y 2m                                                    &        &           &            &                   &  \\ \hline
							15 & Whole Blood & 7m, 11m, 1y 1m                                           & Male   & Caucasian & Breastfed  &                   & Sister with IgA def. \\ \hline
							16 & Whole Blood & 8m, 1y 1m, 1y 4m, 1y 6m                                  & Male   & -         & -          & Vaginal           & -                    \\ \hline
							17 & Whole Blood & 9m, 1y 2m, 1y 6m                                         & Male   & -         & -          & Vaginal           & -                    \\ \hline
							18 & Whole Blood & 7m, 9m, 1y 1m                                            & Female & -         & -          & Caesarian section & -                    \\ \hline
							19 & Whole Blood & 7m, 9m, 11m, 1y 2m, 1y 6m                                & Male   & -         & -          & Vaginal           & -                    \\ \hline
							20 & Whole Blood & 9m, 11m, 1y 2m, 1y 6m                                    & Male   & -         & -          & Vaginal           & -                    \\ \hline
							21 & Whole Blood & 11m, 1y 3m, 1y 5m, 1y 8m                                 & Male   & -         & -          & Caesarian section & -                    \\ \hline
							22 & Whole Blood & 7m, 9m, 11m, 1y 8m                                       & Female & -         & -          & -                 & None                 \\ \hline
							23 & Whole Blood & 9m, 1y 2m, 1y 6m                                         & Male   & -         & -          & -                 & None                 \\ \hline
							24 & Whole Blood & 7m, 9m, 1y 1m, 1y 6m, 1y 11m, 2y 1m                      & Female & -         & -          & -                 & None                 \\
							& Tonsils     & 2y 4m                                                    &        &           &            &                   &  \\ \hline
							25 & Whole Blood & 7m, 11m, 1y 3m, 1y 5m, 1y 8m                             & Male   & -         & -          & -                 & Sister with IgA def. \\ \hline
							26 & Whole Blood & 8m, 1y 1m,                                               & Male   & -         & -          & -                 & None
						\end{longtable}
					\end{landscape}	
%\appendix
	\section{Appendices}
		
		\subsection{Sample information}

			
			
			
			
			\begin{landscape} % Sample table
				\footnotesize
				\begin{longtable}[c]{|l | l |p{4cm}|l|l|l|l|l|}
					\caption{Clinical details of patients with THI. Abbreviations: m, months; y, years; CVI, common variable immunodeficiency; -, unknown; def, d.} \\ \hline 	
					Individual & Tissue & Age at test  & Gender  & Ethnicity   & Breastfeeding status & Mode of birth & Family history of PID    \\ \hline \hline
					\endfirsthead
					\hline
					Individual & Tissue & Age at test  & Gender  & Ethnicity   & Breastfeeding status & Mode of birth & Family history of PID    \\ \hline \hline
					\endhead
					\hline 	\endfoot 
					\label{table:samples}
					1  & Whole Blood & 7m, 11m, 1y 7m, 2y 1m                                    & Female & Caucasian & Breastfed  & Vaginal           & None                 \\
					   & Tonsils     & 2y 1m                                                    &        &           &            &                   &  \\ \hline
					2  & Whole Blood & 8m, 11m, 1y 3m, 1y 6m, 1y 8m, 2y 1m, 2y 4m, 2y 6m, 2y 8m & Male   & Caucasian & Breastfed  & Vaginal           & None                 \\
					   & Tonsils     & 2y 6m                                                    &        &           &            &                   &  \\ \hline
					3  & Whole Blood & 9m, 1y 3m, 1y 9m, 2y 6m, 2y 9m, 3y 2m, 3y 6m             & Male   & Jewish    & Bottle fed & Caesarian section & None                 \\
					   & Tonsils     & 3y 2m                                                    &        &           &            &                   &  \\ \hline
					4  & Whole Blood & 10m, 1y 1m, 1y 8m                                        & Male   & Caucasian & Bottle fed & Vaginal           & None                 \\ \hline
					5  & Whole Blood & 9m, 11m, 1y 3m, 1y 5m, 1y 8m                             & Male   & Asian     & Breastfed  & Vaginal           & Brother with CVI     \\ \hline
					6  & Whole Blood & 7m, 9m, 11m, 1y 8m                                       & Female & Jewish    & Breastfed  & Vaginal           & None                 \\ \hline
					7  & Whole Blood & 9m, 11m, 1y 2m, 1y 6m                                    & Male   & -         & Breastfed  & -                 & -                    \\ \hline
					8  & Whole Blood & 7m, 9m,                                                  & Female & -         & Breastfed  & -                 & -                    \\ \hline
					9  & Whole Blood & 7m, 9m, 11m, 1y 3m 1y 8m                                 & Male   & -         & Breastfed  & -                 & -                    \\ \hline
					10 & Whole Blood & 7m, 9m, 11m, 1y 2m, 1y 6m                                & Male   & -         & Breastfed  & -                 & -                    \\ \hline
					11 & Whole Blood & 9m, 11m, 1y 1m, 1y 3m 1y 8m                              & Male   & -         & Bottle fed & -                 & -                    \\ \hline
					12 & Whole Blood & 7m, 11m, 1y 2m, 1y 11m, 2y 6m, 2y 9m, 3y 4m, 3y 7m, 4y   & Female & Jewish    & Breastfed  & Vaginal           & None                 \\
					   & Tonsils     & 2y 11m                                                   &        &           &            &                   &  \\ \hline
					13 & Whole Blood & 9m, 11m, 1y 2m, 1y 6m, 2y 4m, 2y 8m                      & Male   & Jewish    & Bottle fed & Vaginal           & None                 \\ \hline
					14 & Whole Blood & 9m, 1y 1m, 1y 7m, 2y 2m, 2y 9m, 3y 2m                    & Female & Caucasian & Bottle fed & Vaginal           & None                 \\
					   & Tonsils     & 3y 2m                                                    &        &           &            &                   &  \\ \hline
					15 & Whole Blood & 7m, 11m, 1y 1m                                           & Male   & Caucasian & Breastfed  &                   & Sister with IgA def. \\ \hline
					16 & Whole Blood & 8m, 1y 1m, 1y 4m, 1y 6m                                  & Male   & -         & -          & Vaginal           & -                    \\ \hline
					17 & Whole Blood & 9m, 1y 2m, 1y 6m                                         & Male   & -         & -          & Vaginal           & -                    \\ \hline
					18 & Whole Blood & 7m, 9m, 1y 1m                                            & Female & -         & -          & Caesarian section & -                    \\ \hline
					19 & Whole Blood & 7m, 9m, 11m, 1y 2m, 1y 6m                                & Male   & -         & -          & Vaginal           & -                    \\ \hline
					20 & Whole Blood & 9m, 11m, 1y 2m, 1y 6m                                    & Male   & -         & -          & Vaginal           & -                    \\ \hline
					21 & Whole Blood & 11m, 1y 3m, 1y 5m, 1y 8m                                 & Male   & -         & -          & Caesarian section & -                    \\ \hline
					22 & Whole Blood & 7m, 9m, 11m, 1y 8m                                       & Female & -         & -          & -                 & None                 \\ \hline
					23 & Whole Blood & 9m, 1y 2m, 1y 6m                                         & Male   & -         & -          & -                 & None                 \\ \hline
					24 & Whole Blood & 7m, 9m, 1y 1m, 1y 6m, 1y 11m, 2y 1m                      & Female & -         & -          & -                 & None                 \\
					   & Tonsils     & 2y 4m                                                    &        &           &            &                   &  \\ \hline
					25 & Whole Blood & 7m, 11m, 1y 3m, 1y 5m, 1y 8m                             & Male   & -         & -          & -                 & Sister with IgA def. \\ \hline
					26 & Whole Blood & 8m, 1y 1m,                                               & Male   & -         & -          & -                 & None
				\end{longtable}
			\end{landscape}	
				
	\bibliographystyle{BrittSuperScript}
	\bibliography{../literature}
	
%	\includepdf[pages=-]{../Reading/Transient_hypogammaglobulinemia_of_infancy_Ovadia_and_Dalal_2014.pdf}
%	\includepdf[pages=-]{../Reading/Whole-genome-fingerprint-of-the–DNA-methylome-during-human-B-cell-differentiation.pdf}
	
	
	
\end{document}