\documentclass[12pt]{article}

\usepackage[margin = 2.4cm]{geometry} % For margins of 3cm
\usepackage{float} % For H float position
\usepackage{gensymb} % For some symbols
\usepackage{amsfonts, amssymb, amsmath} % All three for maths symbols
\usepackage[export]{adjustbox} % For figure frames
\setlength{\parskip}{6pt} % To make nice looking paragraph spacing
\usepackage[export]{adjustbox} % For figure frames
\usepackage[numbers, super, sort&compress]{natbib} % bibliographies

\title{Methylation status of B cells of those afflicted with Transient Hypogammaglobulinemia}
\date{Research Hypothesis and Experimental Proposal}
\author{Brittany Howell \\ a1646948}

\newcommand{\CDH}{CD4$^+$ } %Thelper
\newcommand{\CDR}{CD4$^+$CD25$^+$ } %Treg
\newcommand{\IFNG}{IFN-$\gamma$ }
\newcommand{\NFKB}{NF-$\kappa$B }
\newcommand{\naive}{na\"{i}ve }

\begin{document}
	\maketitle
	%	\tableofcontents
	
	\section{Background}
		
		\subsection{THI}
	
		Immunoglobulins are a vital component of the adaptive immune system \citep{Simon15}.  
		The B cells which produce antibodies (active immunoglobulin) in adults are not fully mature in young infants, resulting in a decrease in serum immunoglobulin levels after birth \citep{Martin10, Rechavi15}. 
		Physiologic hypogammaglobulinemia refers to the point when serum immunoglobulin reaches its lowest point, commonly at 4-6 months of age \citep{Dressler89}. 
		In some individuals, immunoglobulin levels are extremely low, due to defects in the immune system \citep{Alherz14}. 
		Transient hypogammaglobulinemia (THI) is one such disorder whereby affected persons have a prolongation or exacerbation of regular hypogammaglobulinemia, followed by spontaneous recovery \citep{Gitlin56,Rosen66,Tiller78,McGeady87,Dressler89,Dalal98,Alherz11,Alherz14}.
		The mechanism causing low serum immunoglobulin in THI patients has not been elucidated.
		
	
	\section{THI diagnosis}
	
	While THI is consistently described as an extension of physiologic hypogammaglobulinemia, the precise antibody levels required for diagnosis are not consistent throughout the literature. 
	Antibodies are produced in many classes including IgG, IgM, IgA and IgE \citep{Cooper15}
	Most THI definitions require that the concentration of at least one class of antibody be at a serum concentration more than two standard deviations (SD) below the mean for age matched controls \citep{Tiller78,McGeady87,Dressler89,Dalal98}. 
	{\Huge May be important, may be able to delete}
	
	
	\subsection{Review of etiology and theories: \citeauthor{Ovadia14} \citeyear{Ovadia14}}
	
	\begin{itemize}
		
		\item The cause of THI remains unknown despite numerous pathogenic mechanisms being proposed.
		\begin{enumerate}
			\item Fundenberg and Fundenberg - demonstrated that gamma globulin antigenic determinants present in human fetal IgG molecules and not in maternal IgG can stimulate an immune response to fetal IgG during pregnancy. They concluded that alloantibodies might cross the placenta and cause transient suppression of fetal immunoglobulin production - prospective study on the issue did not support the hypothesis. 
			\item Soothill suggested that THI is a manifestation of genetic heterozygosity for some other ID diseases - it remains a possibility but an example has not been found 
			\item Siegel proposed a defect in TH cell maturation, following an observation of low T cell numbers, his observations have not been supported by many other studies
			\item More recent work has suggested a role for cytokines in the pathogenesis of THI. An enhanced production of TNF$\alpha/\beta$ and IL-10 has been observed. Adding TNF$\alpha/\beta$ was shown to inhibit IgG and IgA secretion. It was concluded that TNF may be involved in regulating IgG/IgA production and the balance between TNFs which suppress IgG production, and IL-10 which inducges IgG production, may be important for the normal development of IgG secreting B cells.
			\item \citet{McGeady87} mentioned that frequent administration of antibiotics could potentially diminish bacterial gut flora, however a number of patients in the study did not receive antibiotics frequently so it seems improbable.
		\end{enumerate}
		\item The defining factor of THI is that it has a gradual tendency to increase, unlike X linked or common variable agammaglobulinemia - spontaneous recovery is said to occur between 2 and 4 years of age \citep{Tiller78} "Most of the children did well after 2 years of age." \citep{McGeady87}
		\item \citet{McGeady87} said that the children they investigated had fewer infections as they grew older and cited two papers with similar observations
		\item most studies have found that lymphocyte subpopulations and cellular immunity are intact, including the level of memory and class-switched B cells. 
		\item Investigators have found reduced frequencies of both circulating IgM+ and ``switched" memory B cells as well as an inability to produce IgG in vitro \citep{Ovadia14}.
		\item \citet{McGeady87} said that the concept of delayed activation is consistent with most immunologists' current perception of THI
	\end{itemize} 
	
	\subsection{B cell features in transient hypogammaglobulinemia of infancy}
	
	\subsubsection{B cells in hypogammaglobulinemia - \citeauthor{Fiorilli86} \citeyear{Fiorilli86}}
	
	
	
	\begin{itemize}
		\item Immunoglobulin deficiency can be the result of (1) a failure of pre-B cells to differentiate into B cells, as in X-linked hypogammaglobulinemia, (2) a defect of isotype switch or (3) a failure of B cells to differentiate terminally into antibody-secreting plasma cells.
		\item (3) is commonly seen in patients with common variable immunodeficiency, which may present with low, normal or increased numbers of circulating B cells carrying IgM, IgG or IgA molecules on their surface.
		\item A number of studies have indicated that distinct mechanisms may prevent patients' B cells from differentiating properly, including the so-called ``intrinsic B cell defects", deficiencies, deficiencies of helper T cells, and the presence of activated autosuppressive T cells
		\item B cells of some patients with CVI have been said to display patterns of membrane immunoglobulin isotypes resembling those of immature B lymphocytes
		\item Study investigated the presence of immature B cells, ie, cells carrying simultaneously IgG and IgM molecules on their surface, in 12 patients with primary IgG deficiencies. 
		\item In 10 patients (except 2 with CVI) have the majority of the circulating IgG bearing B cells also expressing surface IgM. 
		\item Found that patients with a profound deficiency of serum IgG usually have significant numbers of circulating IgG bearing B cells. - In most cases they resembled immature B lymphocytes in that they express multiple surface immunoglobulin isotypes
		\item Platts-Mills et al. found that B cells from the majority of CVI patients behaved as functionally immature cells in the sense that they produced IgM but very little IgG or IgA in vitro. The other CVI patients had CVI which was uniformly associated with autoimmune disorders or thymomas as well as increased suppression, suggesting a secondary (acquired) hypogammaglobulinemia. Furthermore, B cells from patients of the latter group responded to polyclonal activation with Epstein-Barr virus in a manner qualitatively similar to that of mature B cells. 
	\end{itemize}
	
	\subsubsection{Subsets of transient hypogammaglobulinemia of infancy \citeauthor{Dalal98} \citeyear{Dalal98}}
	
	\begin{itemize}
		\item Study aimed to characterise the subsets of THI, in an attempt to define the disorder. 
		\item Used 35 patients, assigned to three categories:
		\begin{enumerate}
			\item Patients eventually have normal total serum levels with normal IgG subclass division, and normal specific antibody production. The process may have transient phase whereby some IgG subclasses may become unbalanced 
			\item Patients continue to have low serum IgG levels, and poor antibody titres
			\item Serum IgG becomes normal, but individuals remain incapable of mounting an adequate antibody response
		\end{enumerate}
		\item The groups are likely to represent a heterogenous group of different genotypes, and it is important to understand the phenotypes, before the genotypes are explored
	\end{itemize}
	
	\section{Lineage commitment and methylation status in other cells}
	
	{\bf Epigenetic changes have been coined as the hallmark of cell differentiation}. T cell activation and skewing, a certain type of cell differentiation, is governed in great parts by epigenetic changes which insure that the clone of a T cell will retain its phenotype (Th1/Th2 etc).
	
	As said by \citet{Choukrallah14} the same TFs can be equally expressed in different cell types, and yet have different binding profiles. So, addition to the DNA sequence recognition, TF binding strongly depends on chromatin structure and epigenetic modification.
	
	\subsection{Methylation}
	
	\begin{itemize}
		\item DNA methylation was the first epigenetic mechanism recognised, and the one that is most extensively studied. \citep{Begin14}
		\item \textit{De novo} methylation occurs in response to various cellular stressors, and results in the addition of a methyl group to position 5 of a cytosine residue \citep{Begin14}.
		\item CpG islands have clusters at promoters and enhancers. Up to 90\% of genome CpG sites are methylated, with most unmethylated CpG islands being in active genes \citep{Begin14}
		\item \citet{Schmidl09}: Methylation of cytosine residues in genomic DNA is an important epigenetic mark that is essential for normal embryonic development in mammals, imprinting, X inactivation and silencing of potential hazardous genetic elements such as transposons
		\item DNA methylation provides an additional mechanism for gene regulation, it can occur at the fifth position of cytosine and can repress via the following mechanisms: (1) Inhibit protein binding (like TFs). (2) Recruit proteins containing domains which interfere with transcription by recruiting repressors \citep{Choukrallah14, Begin14}. 
	\end{itemize}
	
	\subsection{T cells}
	
	\subsubsection{\citeauthor{Begin14}\citeyear{Begin14}: T cell methylation in development}
	
	\begin{itemize}
		\item The Th1/Th2 balance of T cells can be affected by environmental exposures which change epigenetic controls.
		\item In resting \CDH cells, the IL-4 and \IFNG genes are methylated 
		\item Upon allergic sensitisation, the IL-4 promoter in allergen-specific T cells is demethylated, correlating with IL-4 expression.
		\item The IL-4 locus of Th2 cells is also marked with permissive histone modifications (H3K4me3) which are absent in Th1 or na\"{i}ve T cells.
		\item At the main Th2 gene locus on chromosome 5, a chromatin hub that interacts with GATA-3 is formed.
		\item GATA-3 interacts with HAT enzyme p300 and Chd to induce permissive histone and chromatin changes.
		\item GATA-3/Chd complex binds HDAC to repress the locus encoding TBET, the master regulator which activates Th1 and suppresses Th2 genes
		\item Further suppression of Th1 cytokines is achieved by the increase of their DNA methylation from na\"{i}ve state. 
		\item Th2 genes are demethylated - mechanism is still incompletely understood.
		\item Epigenetics is complicated: the GATA-3 promoter has been shown to keep its repressive histone modification despite Th2 activation. It presents a bivalent state with repressive and activating histone modifications - positive feedback is important to insure stable expression.
		\item In the thymus, iTregs are induced by TCR engagement, subsequent \NFKB signalling induces permissive histone modifications and potentially initiates chromatin remodelling in the Foxp3 locus.
		\item DNA demethylation of the CNS2, also called the Treg-specific demethylated region (TSDR) is a major event in tTreg differentiation, and carries an important function in Foxp3 stabilising Foxp3 expression. CNS2 is the site at which Foxp3 binds to its own gene to maintain expression in a positive feedback mechanism allowing for a persistent phenotype and suppressive function
		\item Activated, Non Treg T cells also express Foxp3 upon TCR engagement, however the expression is transient owing to the continued methylation of the CNS2. 
		\item Comparing tTregs to Foxp3$^+$ activated effector T cells, there are hundreds of loci throughout the genome which show demethylation and correspond to binding sites for Foxp3. The methylation changes are not induced by Foxp3, but rather, allow Foxp3 to access its targets and exert its function. 
		\item In activated T cells, the loci are methylated, which could explain the difference in function despite the expression of Foxp3 in activated T cells.
	\end{itemize}
	
	\subsubsection{\citeauthor{Schmidl09} \citeyear{Schmidl09}: Differentially methylated regions that may be enhancers}
	
	Study used conventional CD4$^+$ T cells to compare to CD4$^+$ CD25$^+$ T regs. Differentially methylated regions were found and thought to indicate methylation sensitive enhancers
	
	\begin{itemize}
		\item It has been shown that the functional program of Treg cells is at least partially controlled by the miRNA pathways 
		\item For continuous expression of the lineage-directing transcription factor Foxp3, the methylation status of a methylation-sensitive, Treg cell-specific enhancer in intron I is important.
		\item The restriction of cell type-specific enhancers is a key function of DNA methylation. 
		\item STAT5:
		\begin{itemize}
			\item Regions specifically demethylated in Treg cells were enriched for STAT5 consensus sites. 
			\item TF STAT5 is activated by IL-2
			\item Treg cell survival requires IL-2
			\item STAT5 also has an essential role in Treg homeostasis and is known to regulate Foxp3 through an intronic, methylation-sensitive enhancer. 
			\item It would make sense that STAT5 bind at this region, but it has not been confirmed \textit{in vivo}
		\end{itemize}
		\item In T-reg cells, A functionally important intronic enhancer of the Foxp3 gene was shown to be methylation sensitive properties.
		\item In the study, conventional CD4$^+$ T cells were compared to CD4$^+$ CD25$^+$ T regs
		\item Half of the tested DMRs (differentially methylated regions) significantly enhanced the activity of a heterologous promoter in transient reporter gene assays performed in a T cell leukaemia line.
		\item All regions lost enhancer activity upon CpG methylation.
		\begin{itemize}
			\item Eg. Found methylation sensitive enhancer in intron 4 of CD40LG in Tconv (CD4$^+$ cells). CD40L is important in regulating B cell function through interaction with CD40 on B cells and Dendritic cells.
		\end{itemize} 
	\end{itemize}
	
	\section{B cell development and differentiation}
	
	\subsection{Stages and markers of B cell differentiation:}
	
	\begin{itemize}
		\item Pro-B cells express B220, which coincides with entry into B cell lineage
		\item pre-BI cells express CD19 and complete recombination of heavy chain IgH D to J segments. 
		\item Next stage sees generation of IgH V(D)J alleles, allowing heavy chain expression which assembles with the surrogate light chain to form the pre-B cell receptor.
		\item Cells need to pass functional tests here
		\item small pre-BII cells rearrange the light chain allowing formation and exposure of a functional Ig molecule (BCR)
		\item immature cells can leave the bone marrow and enter the periphery \citep{Choukrallah14}]
	\end{itemize}
	
	\subsection{Transcription factors in B cell development}
	
	\subsubsection{Pu.1}
	
	\begin{itemize}
		\item Very upstream TF, essential for the development of lymphoid cells as well as macrophages and neutrophils. 
		\item Disruption of PU.1 in mouse was shown to prevent commitment of MPPs towards lymphoid lineage
	\end{itemize}
	
	\subsubsection{IKAROS}
	
	\begin{itemize}
		\item Also upstream
		\item Mutational disruption of Ikaros DNA-binding domain leads to an early block in lymphopoiesis before commitment to lymphoid restrictd stages.
		\item Also involved in later stages of B cell development, where it promotes heavy chain gene rearrangement by inducing expression of RAG1/2 genes.
		\item Also required for differentiation of large pre-B cells to small pre-B cells and for transcription and rearrangement of the IgL locus.
	\end{itemize}
	
	\subsubsection{E2A}
	
	\begin{itemize}
		\item Required for Ebf1 and FoxO1 expression at the Common Lymphoid Progenitor stage
		\item E2A mutant mice lack B cells
	\end{itemize}
	
	\subsubsection{EBF1}
	
	\begin{itemize}
		\item Essential for B cell specification and commitmnt. 
		\item Regulates expression of genes required for B cell development including FoxO1 and Pax5
	\end{itemize}	
	
	\subsubsection{Pax5}
	
	\begin{itemize}
		\item Essential for B cell commitment and maintenance of B cell identity through activation of B cell specific genes and repression of lineage inappropriate genes
		\item Deletion of Pax5 in mature B cells leads to de-differentiation to lymphoid progenitors, which can differentiate into functional T cells
	\end{itemize}
	
	\subsubsection{FoxO1**}
	
	\begin{itemize}
		\item Early deletion of FoxO1 causes substantial block at pro-B cell stage due to failure to express IL-7 receptor alpha chain 
		\item Inactivation of FoxO1 in late pro-B cells results in arrest at pre-B cell stage due to impaired expression of RAG1/2 (direct targets of FoxO1)
		\item Deletion in peripheral B cells leads to reduced number of LN B cells, due to down regulation of L-selectin and defect in class-switch recombination
	\end{itemize}
	
	\subsubsection{c-Myb and Runx}
	
	\begin{itemize}
		\item Deletion of c-Myb in mice leads to a block at the pre-pro B cell stage which is accompanied with impaired expression of the $\alpha$ chain receptor and Ebf1
		\item Deletion of Runx1 also causes a developmental block at the the pro-B cell stage accompanied by reduced expression of E2A, Ebf1 and Pax5. 
		\item Runx1-deficient pro-B cells were shown to harbour excessive amounds of the repressive histone mark H3K23me3 in the Ebf1 proximal promoter
		\item Retroviral transduction of Ebf1, not Pax5, into Runx1-deficient progenitors restores B cell development. 
	\end{itemize}
	
	\subsection{Transcription factors in B cell differentiation - \citeauthor{Li13} \citeyear{Li13}} 
	
	\begin{itemize}
		\item Resting B cells display genome wide DNA hypomethylation
		\item Genes crucial for the maintenance of B cell identity (\textit{Pax5, Spib, Ebf1}) and B cell marker genes (\textit{CD19}), display active epigenetic state
		\item Chromatin in transcribed \textit{Igh} V$_H$DJ$_H$ regions, the intronic $\mu$ enhancer and the \textit{Igh} 3' locus control region contains hypomethylated DNA and activating histone modifications
		\item The epigenetic marks were likely introduced during B cell development, because the open chromatin state of these regions is required for V(D)J recombination
		\item Active epigenetic marks in the \textit{Igh} locus and in the \textit{Pax5, Spib, Ebf1} and \textit{CD19} loci persist during na\"{i}ve B cell activation
		\item Upon activation by antigens, B cells undergo DNA demethylation and histone modifications, and express a specific set of miRNAs.
		\item Repression of the \textit{Aicda} gene in \naive B cells is mediated by promoter hypermethylation, during B cell activation, \textit{Aicda} DNA is demethylated and the locus becomes enriched in active histone modifications
		\item DNA hypomethylation seems to be important, as B cells carrying identical pre-rearranged Ig$\kappa$ alleles, only the hypomethylated allele is hypermutated despite comparable transcription of both alleles.
		\item S regions: the genes which contain IgG, IgA etc genes. They are acted upon when undergoing Class Switch Recombination
		\item Active epigenetic state is found in even \naive B cells, indicating that S$\mu$ is in a constitutively open state, poised for switching.
		\item For plasma cell differentiation: Blimp-1 (encoded by \textit{Prdm1}. Epigenetic induction of Blimp-1 causes events which drives plasma cell differentiation and possibly maintains plasma cell identity
		\item Differentiating into memory cells not likely a problem
		\item Overall DNA hypomethylation has been associated with systemic autoimmune diseases
	\end{itemize}
	
	\begin{table}[h]
		\begin{tabular}{l l l}
			Region/gene         & Epigenetic mark     & function of epigenetic mark                           \\
			\hline
			V(D)J               & DNA hypomethylation & Increases region accessibility                        \\
			\textit{Igh} 3' LCR & DNA hypomethylation & Mediates germline VDJ and I$_H$-S-C$_H$ transcription 
		\end{tabular}
	\end{table}
	
	\subsection{DNA methylation in B cells during maturation and differentiation}
	
	\subsubsection{Why epigenetics?}
	
	\begin{itemize}
		\item \citet{Lara14} noted that mutations in loss of chromatin factors lead to haematopoiesis defects and disease
		\item \citet{Tagoh04} said that even before the onset of gene expression and stable TF binding, specific chromatin alterations are observed (including methylation changes). Hence the idea that epigenetic programs guiding blood cell differentiation are engraved into the chromatin of lineage-specific genes, and such chromatin changes are implemented \textbf{before cell lineage specification}; Epigenetic programs are engraved into the chromatin of lineage-specific genes before cell lineage specification and the onset of detectable gene expression
		\item Differentiation and lineage commitment are associated with specific methylation or demethylation events \citep{Schmidl09}
	\end{itemize}
	
	\subsubsection{Methylation status in B cells}
	
	\begin{itemize}
		\item Methylation loss is observed as B cells mature \citep{Oakes16}.
		\item Hypomethylation is enriched in enhancer/promoter regions
		\item The TF families which show hypomethylation are AP-1, EBF, RUNX, OCT, IFF and NF$\kappa$B
		\item The cell subtypes which show the most pronounced methylation changes in comparison to the preceding stage are germinal centre B cells, memory B cells and BM plasma cells \citep{Kulis15}
		\item It is possible to accurately classify B cells into their maturation stage by the methylation state of 5 CpGs in genes important to B cell differentiation \citep{Kulis15}.
		\item Transition from HPCs to pre-B1 cells has an inverse corelation between the expression of TFs and the methylation of their binding sites; High methylation status occurs with low expression of TFs \citep{Kulis15}. 
	\end{itemize}
	
	
	\section{Proposal}
	
	
	
	\begin{itemize}
		\item Bisulfite sequencing of enhancers or transcription factors that are important to B cell development
		\item Take B cells from infants with CVID and THI, as well as controls. B cells should be taken at different stages, up until age 5
	\end{itemize}
	
	\newpage
	%	\bibliographystyle{brittany-superscript}
	\bibliographystyle{bens2}
	\bibliography{../literature}
	
\end{document}