\documentclass[12pt]{article}
\usepackage[margin = 2.4cm]{geometry} % For margins of 3cm
\usepackage{gensymb} % For some symbols
\usepackage{amsfonts, amssymb, amsmath} % All three for maths symbols
\usepackage[export]{adjustbox} % For figure frames
\setlength{\parskip}{6pt} % To make nice looking paragraph spacing
\usepackage[export]{adjustbox} % For figure frames
\usepackage[numbers, super, sort&compress]{natbib} % bibliographies
\usepackage{setspace} % Allows double spacing
\usepackage{pdfpages} % Allows including PDFs
\usepackage{longtable} % Allows longtables
\usepackage{lscape} % To allow some pages to be in landscape
\usepackage{float, subfloat} % For H flag and subfloats
\usepackage{caption, subcaption} % for ABCD Subcaptions
% Longtable settings
\setlength\LTleft{0pt} % flush left
\setlength\LTright{0pt} % flush right
\setlength{\LTcapwidth}{8in}


\doublespacing % Makes all lines (not figure legends) double spaced

\title{\vspace{-2cm} Transcriptome and epigenome profile of B cells in patients with Transient Hypogammaglobulinemia of Infancy}
\date{}
\author{Brittany Howell}

\newcommand{\CDH}{CD4$^+$ } %Thelper
\newcommand{\CDR}{CD4$^+$CD25$^+$ } %Treg
\newcommand{\IFNG}{IFN-$\gamma$ }
\newcommand{\NFKB}{NF-$\kappa$B }
\newcommand{\naive}{na\"{i}ve }

\newcommand{\Th}{T$_\textrm{H}$ }
\newcommand{\ThO}{T$_\textrm{H}1$} 
\newcommand{\ThT}{T$_\textrm{H}2$ }
\newcommand{\Tc}{T$_\textrm{C}$ }

%%%%%%%%%%%%%%%%%%%%%%%%%%%%%%%%%%%%%%%
%%%%%%%%%  Word count: 2500  %%%%%%%%%%  
%%%%%  Inc. all legends & titles  %%%%%  
%%%%%%%%%%%%%%%%%%%%%%%%%%%%%%%%%%%%%%%


\begin{document}
	\maketitle
	
	\paragraph{Hypothesis}
	~\\
	Patients with Transient Hypogammaglobulinemia of Infancy (THI) will exhibit delayed loss of methylation in B cell lymphopoiesis genes, resulting in a deficiency of mature B cell subpopulations.
	
	\section{Background}
		
		Antibodies are a vital component of the adaptive immune system \citep{Simon15}. 
		The production of antibodies occurs when \naive B cells are activated in response to foreign antigens (figure \ref{fig:BCellDevelopment}).
		During gestation, maternal antibodies (mAbs) are passed to the foetus through the placenta \citep{Hasselquist09}.
		At birth, B cells do not fully mature, so as mAbs are degraded, serum antibody concentration decreases \citep{Martin10,Rechavi15}. 
		Physiologic hypogammaglobulinemia refers to the point when serum antibody concentration reaches its lowest level.
		In normal infants, the hypogammaglobulinemia subsides as the B cells are able to mature into antibody producing cells, commonly at 4 to 6 months of age \citep{Dressler89}. 
		Transient Hypogammaglobulinemia of infancy (THI) is a disorder where regular hypogammaglobulinemia is prolonged or exacerbated then spontaneously alleviated. \citep{Dressler89,AlHerz14,Gitlin56,AlHerz11,Rosen66,McGeady87,Stiemh80, Dalal98,Tiller78,Buckley83}.
		The mechanism causing low serum immunoglobulin in THI patients is unknown \citep{AlHerz14}. 

			
		\paragraph{Cause of THI} 
			~\\
			Studies investigating THI have found that levels of circulating B cells are normal and subpopulations of B cells are intact \citep{Tiller78,Stiemh80,Siegel81,Buckley83,Fiorilli86,Dressler89}.
			With no obvious B cell deficiency, the cause of THI has been speculated extensively, but no proposed cause has been supported by replicated evidence \citep{Fudenberg64,Rosen66,Nathenson71,Willenbockel60,Soothill68,Tiller78,Fiorilli86,Ovadia14,Siegel81,McGeady87}.
			In regards to genetic inheritance, THI was initially thought to be familial \citep{Willenbockel60}. 
			\citet{Soothill68} proposed that THI was a manifestation of genetic heterozygosity for other immunodeficiency diseases, noting the high number of patients who had immunodeficient relatives.
			While it remains a possibility, no proceeding studies have shown supporting evidence \citep{Tiller78,Fiorilli86, Ovadia14}.
			With no obvious deficiencies or genetic links, THI is most often speculated to be caused by some kind of delay in B cell maturation or activation \citep{McGeady87,Stiemh80,Walker94,Rosen84}. 
			
		\paragraph{B cell development}
			~\\
			Immunoglobulin deficiency can result from B cell precursors failing to develop into mature B cells or mature B cells failing to differentiate into antibody secreting plasma cells \citep{Fiorilli86}. 
			All antibodies are produced by mature B cells, developed from haematopoietic stem cells in a pathway shown in figure \ref{fig:BCellDevelopment}.
			First, the hematopoietic stem cells differentiate into common lymphoid progenitors, which commit to the B cell lineage.
			B cells gradually rearrange their immunoglobulin genes and differentiate into mature \naive B cells which leave the bone marrow to enter the periphery. 
			Resting naive B cells transit through lymph nodes where they encounter specific antigen, activating them and inducing the germinal centre reaction. 
			Further rearrangement of immunoglobulin genes occurs followed by rapid proliferation and differentiation into plasma and memory cells. 
			Plasma cells produce the antibodies required for humoral immunity. 
			The isotype (IgM/G/A/E) of the antibodies is determined by the environmental signals present at the activated B cell stage. 
			
			
			\begin{figure}[tb]
				\centering
				\includegraphics[width=\linewidth]{../Figures/Bcelldevelopment.png}
				\caption{B cell development from hematopoietic stem cell to memory and plasma cells. Phases are shown as antigen independent or dependent and location (bone marrow or periphery) is indicated. IGM/G/A/E indicates option of antibody isotype. BCR: B cell receptor.}
				\label{fig:BCellDevelopment}
			\end{figure}
			
			B lymphocyte differentiation is tightly regulated by transcription factors (TFs) functioning in a complex network of auto-regulation, cross regulation, and positive and negative feedback loops.
			The transcriptional regulation of B cell development is not a single hierarchical cascade, many transcription factors work cooperatively to direct the regulation and expression of genes and other TFs (figure \ref{fig:TFBcell}). 
			Transcription factors prominent in early stages of B cell specification and commitment include E2A, Ebf1, Pax5 and FoxO1. 
			They influence B cell development by promoting changes in chromatin remodelling, facilitating DNA methylation or demethylation and interacting with other factors \citep{Gao09,Maier04,Walter08,Decker09,Lin10,McManus11,Treiber10,Zandi00}. 
			Ikaros, PU.1, E2A and FoxO1 are also involved in lineage fate determination, but are not restricted to the B cell lineage.
			Loss of function studies involving the aforementioned TFs show some defect in B cell lineage commitment, resulting in a loss of a B cell subset or the entire B cell lineage \citep{Choukrallah14}.

			
			\begin{figure}
				\centering
				\includegraphics[width = 0.4\linewidth]{../Figures/BcellTF.png}
				\caption{Schematic representation of interactions between early B cell transcription factors. Adapted from \citet{Choukrallah14}.}
				\label{fig:TFBcell}
			\end{figure}
			
			For the first few months of life, two distinct types of B cells are produced. 
			The aforementioned B cells are conventional B cells, B2 cells, which are the predominant subclass for most of life \citep{Simon15}. 
			For the first few months of life however, B1 cells comprise up to 40\% of peripheral blood B cells \citep{Griffin11}. 
			B1 cells spontaneously secrete low-affinity IgM with a limited range of antigen specificities offering a first line of defence with more similarity to an innate response.
			The low diversity exhibited by B1 cells is attributed to occurrence of fewer somatic mutations \citep{Montecino-Rodriguez12}.
			B1 cells have predominant importance in the first few months of life, so it is important to consider them in context of THI. 
			However, THI is characterised by a lower than normal serum IgG level, IgG is produced predominantly by conventional B2 cells, so while B1 cells may interact with the B cell response, they will not form the focus of this investigation.
			
			
			
		\paragraph{FACS}
			~\\
			Previous investigations into THI have claimed that subpopulations of B cells are the same in THI patients as controls \citep{Tiller78,Stiemh80,Siegel81,Buckley83,Fiorilli86,Dressler89}. 
			The claim is supported by rosette-formation and single or two-colour membrane immunofluorescence studies. 
			The immunofluorescence studies distinguished only between mature and immature B cells, using the observed combinations of surface IgM and IgG. 
			Analysis techniques have improved immensely since 1989, when the most recent study took place.
			Flow cytometry (FACS) in particular has developed extensively, allowing the measurement of an increasing number of parameters per cell \citep{Saeys16}.
			In a FACS analysis, cells are stained with fluorochrome-conjugated antibodies that bind to cell surface markers or intracellular targets. 
			Cells are then placed into a flow cytometer which passes cells individually through lasers.
			The light emitted by the lasers excites the fluorochromes on the cell which produces a signal proportional to the concentration of the target \citep{Aghaeepour13}.
			18-parameter flow cytometry is now routinely used, 30-parameter flow cytometers are becoming commercially available, and 50-parameter flow cytometry is predicted to be available soon \citep{Saeys16}.
			The number of parameters measurable per cell has increased dramatically since 1989, and thence, the potential to distinguish B cell subpopulations is now immensely greater.

			
		\subsection{Lineage commitment}
			
			Epigenetic modifications act in concert with transcription factors to confer the phenotype of many cell subsets of the immune system \citep{Lara14,Zan15,Mercer11}. 
			A prominent example is the activation and differentiation of the many T cell subsets \citep{Begin14, Zeng13}. 
			Confirmation of the T regulatory cell lineage relies on the methylation status of the Treg-specific demethylated region (TSDR) is imperative in Treg differentiation \citep{Polansky08}.
			{\Large Add a sentence about how TSDR is FOXP3 gene}
			When the T cell is stimulated by binding at the T cell receptor, demethylation occurs at the TSDR.
			Foxp3, a protein expressed during T cell activation, is able to bind to the TSDR when methylated. 
			The binding stimulates the expression of Foxp3 and hence steers differentiation to the Treg lineage.
			If the TSDR is methylated, as in non Treg subsets, Foxp3 cannot bind and hence its expression is transient \citep{Ohkura13}. 
			Therefore, demethylation permits FOXP3 binding and thence confirms Treg lineage.
			
			Aberrant epigenetics have recently been implicated as the cause of common variable immunodeficiency (CVID), a disease similar to THI \citep{Tallmadge15}.
			CVID is a late-onset primary immunodeficiency characterised by dysfunction or loss of B lymphocytes, blockage of B cell development at pro-B cell stage and decreased immunoglobulin production. 
			Diagnosis most often occurs between the ages of 20 and 40 years, with patients presenting with recurring bacterial infection \citep{Cunningham-Rundles12}.
			RNA-Seq analysis identified 103 genes which were differentially expressed between healthy controls and CVID patients \citep{Tallmadge15}.
			The most severely down-regulated gene was the transcription factor Pax5.
			Pax5 is essential to commit a cell to the B cell identity through activation of 170 B cell specific genes and repression of at least 110 lineage inappropriate genes \citep{Schebesta07,Delogu06,Roessler07}.
			\citet{Tallmadge15} hypothesised that Pax5 is silenced by aberrant epigenetic mechanisms in lymphocyte progenitors. 
			Epigenome analysis revealed that the Pax5 enhancer was hypermethylated.
			Silencing induced by enhancer methylation would result in a decline in B lymphopoiesis in the bone marrow, followed by a depletion of B cells.
			Furthermore, that the methylation is epigenetic substantiates the late-onset nature of CVID. 
			The prolonged antibody deficiency exhibited in THI patients could be the result of dysfunctional B cells. 
			An epigenetic cause such as methylation 
	
%	\section{Proposal}
%
%		
%			The most intriguing feature of THI is its self-limited nature: recurrent infections gradually subside and serum IgG levels increase with no obvious cause  \citep{Tiller78,Soothill68,Siegel81,McGeady87,Dressler89,Kowalczyk97,Dalal98}. 
%			Furthermore, the lack of evidence supporting a genetic basis suggests that the cause of THI is not within the genome \citep{Tiller78,Fiorilli86,Ovadia14}.
%			
%		
%			In common variable immunodeficiency, a disease related to THI, some B cells resemble immature B cells producing very little IgG \citep{Fiorilli86}. 
%			Incomplete maturation results in limited IgG production, so it is possible that the delayed onset of IgG synthesis in THI is also due to incomplete B cell maturation.
%			Development and differentiation are greatly influenced by epigenetic changes; latent maturity could be caused by inappropriate methylation of B cell development or differentiation genes. 
%	
%			To study incomplete lineage commitment B cells will be sampled from THI patients and age-matched controls then characterised using whole-genome bisulfite sequencing.
%			As in \citet{Kulis15} DNA methylation maps will be generated for sorted human B cell populations: uncommitted haematopoietic progenitor cells, pre-BII cells, plasma cells from bone marrow, germinal center B cells, \naive B cells from peripheral blood and memory B cells from peripheral blood.
%			Global demethylation normally occurs as B cells mature \citep{Oakes16}. 
%			If methylation is a cause of delayed maturation, the methylome of THI patients should be distinct to the age-matched controls.
			
	\section{Experimental aims}
		
		\paragraph{Experimental aim 1:} Using FACS, describe the B cell subpopulations in THI patients and normal individuals throughout early development.
		
		
		\paragraph{Experimental aim 2:} Using whole genome bisulphite sequencing, identify regions of the genome which are differentially methylated in THI samples and controls.
		
		\paragraph{Experimental aim 3:} Using RNA-Seq, identify differentially expressed regions in THI samples.

	\section{Data collection and management}
	
		\subsection{Collecting samples}
			
			% Add topic sentence
			Peripheral blood B cells will be obtained from the buffy coat (figure \ref{fig:BuffyCoat}) of the THI samples, and controls. 
			Plasma cells, germinal centre B cells and \naive B cells will be isolated from processed tonsil samples. 
			Progenitor cells, pre-BI cells, pre-BII cells, immature B cells and plasma cells will be isolated from bone marrow aspirations
			Members from the Pediatric Department at Wolfson Medical centre have kindly agreed to provide whole blood and tonsil samples they have collected from 26 THI patients (see table \ref{table:samples}).
			Further peripheral blood, bone marrow and tonsil samples will be obtained from the Women's and Children's Hospital Immunodeficiency clinic. 
			Blood will be taken from THI patients between initial presentation and normalisation of IgG levels.
			Bone marrow will be taken first upon initial presentation, then in six monthly intervals following, until normalisation of IgG levels
			Control samples will come from donations by healthy subjects. 
			A full clinical record will be kept for every control and experimental sample collected to allow the best possible matching of THI patients with healthy controls. 
		
			\subsubsection{Matching controls} 
			
				To determine B cell maturation changes caused by THI, it is imperative to compare samples that are at the closest possible environmental stage.
				Maturation of B cells integrates numerous internal and environmental signals, so controlling for developmental stage has significant challenges. 
				Where possible, THI samples will be matched to control samples. 
				The most important criteria to match will be age, as it is the most prominent indicator of immune system development. 
				Secondly, the method of both birth (vaginal or caesarian) and feeding (bottle or breastfed) will be considered, as both have large influence over the immune system \citep{Jakobsson14,Cho13,Brandtzaeg03,Rogier14,Gomez14}.
				Further factors such as gender and ethnicity will be matched if possible. 
				Finally the diet of patients will be standardised to reduce the environmental effect on B cell maturation.
				In the best scenario, all of the above criteria will be matched between control and experimental samples. 
				However, there are limitations in the scope of the matching. 
				In the samples provided, there are already gaps in the clinical history of samples (table \ref{table:samples}). 
				Without information such as ethnicity or mode of birth, it is impossible to match a sample to an appropriate control. 
				Additionally, the control samples will be taken from participants' donations which may not match the clinical history of the THI samples.
				Even if it is possible to match samples according to stringent criteria, there are a myriad more external factors that will affect the dynamic nature of the immune system.
				 
				
		
		\subsection{Segregation of B cells}
			
			The maturation of B cells is a continuous process. 
			It is therefore important to ensure that any changes which are observed between experimental and control samples are due to true THI effects, not normal B cell maturation changes.
			To reduce the variation between compared cells, the B cells will be segregated into developmental stages. 
			Preparation and segregation of B cells will be undertaken using FACS as in \citet{Kulis15} and \citet{Oakes16}.
			Subpopulations obtained will include \naive B cells from peripheral blood and tonsils, germinal center B cells and plasma cells from tonsils, and memory B cells from peripheral blood samples (figure \ref{fig:BCellSorting}). 
			There are always limitations when applying discrete developmental stages to continuous processes.
			However, segregating B cells into such specific subtypes significantly reduces the chance of detecting a differences which are due to developmental stage. 
			
			
			
				\begin{figure}[tb]
					\centering
					\begin{subfigure}[b]{.7\textwidth}
						\flushleft
						\includegraphics[width=\textwidth]{../Figures/BcellSorting4}
						\includegraphics[width=.78\textwidth]{../Figures/BcellSorting3}
						\caption{B cell populations}
						\label{fig:BCellSorting}
					\end{subfigure}
					\begin{subfigure}[b]{0.29\textwidth}
						\centering
						\includegraphics[width=\textwidth]{../Figures/BuffyCoat}
						\caption{Buffy coat}
						\label{fig:BuffyCoat}
					\end{subfigure}
					\caption{\ref{fig:BCellSorting}: Description of the FACS sorting markers used to segregate B cell populations. \ref{fig:BuffyCoat}: The buffy coat is the fraction of an anticoagulated blood sample which contains leukocytes and platelets.}
				\end{figure}`
	
	\section{Aim 1:}
	
		\paragraph{Hypothesis} THI samples will be deficient in activated B cells and plasma cells. 
		
		\subsection{Proposed experiment}
			
			Conduct FACS using these antibodies. 
			To show these abundances. 
		
		\subsection{Possible outcomes and interpretations}		
		
			It is possible that THI patients have fewer plasma cells than normal populations. 
			This would lead to a lesser ability to produce antibodies. 
			It would not have been previously noted, as the experiments were not around to show it.
	
	
	\section{Aim 2:}
	
		\paragraph{Hypothesis:} DNA of B cells from THI patients will have regions which are differentially methylated to normal individuals.
		
		\subsection{Proposed experiment}
		
			To investigate global methylation of THI and control samples, we will produce full methylomes of the B cell lineages in figure \ref{fig:BCellSorting}. 
			Whole genome bisulfite sequencing will be used such that we can obtain base-pair resolution of all methylated cytosines within the genome. 
			To perform the analysis, we will use two sets of biological replicates for each of the {\Huge Number} of samples. 
			Samples will undergo two rounds of bisulfite conversion to ensure a cytosine to thymine conversion rate of over 99\%.
			Treated samples will then be sequenced on an Illumina HiSeq 2000 platform and mapped to the genome using the STAR algorithm (v2.4.2a) \citep{Dobin13}.
			
			We will then use ChIP-seq data from the ENCODE project \citep{ENCODE-Project-Consortium12} to analyse methylation status in the context of transcription factor binding sites.
			The relative enrichment of each TFBS in any differentially methylated regions will be calculated in comparison to background reads. 
			A Fisher's exact test will be used to assign an odds ratio and \textit{P} value to each comparison. 
			Of particular focus will be the genes which are specific to B lymphopoiesis such as those in figure \ref{fig:TFBcell}.

		\subsection{Possible outcomes and interpretations}
		
			There are enumerable ways to present the methylation data, it is important to choose a layout which conveys the changes between THI samples and controls, while acknowledging any development related changes.
			One method is to produce unbiased DNA methylation maps of the sorted cell populations as in figure \ref{fig:MethMap} produced by \citet{Kulis15}. 
			The map clearly shows the gradual change in methylation exhibited by the cell subsets. 
			Alternatively, the data could be displayed as in figure \ref{fig:TheoMethMap}, with one map per B cell subset, comparing THI with controls in each figure. 
			An approach which has direct comparison with THI and control samples may make seeing changes easier. 
			A slightly different presentation could be in the form of figure \ref{fig:PlotMeth}, which groups samples by cell subset and allows easy comparison between THI and control samples.
			The most important factor in choosing a method of data presentation, is to find a method which allows real differences to be seen. 
			Due to the dynamic nature of the immune system, it will be a challenge to ensure that only correctly matched samples are compared, but it is imperative for accurate analysis. 
			
			\begin{figure}[tb]
				\centering
				\makebox[\textwidth][c]{
				\begin{subfigure}[b]{0.65\textwidth}
					\flushleft
					\includegraphics[width=1\linewidth]{../Figures/methylomeMap}
					\caption{Methylome map produced by \citet{Kulis15}. }
					\label{fig:MethMap}
				\end{subfigure}
				\begin{subfigure}[b]{0.5\textwidth}
					\includegraphics[width=1\linewidth]{../Figures/MethMapSingle}
					\caption{Theoretical methylome map}
					\label{fig:TheoMethMap}
					\includegraphics[width=\linewidth]{../Figures/MedCpGMeth}
					\caption{Theoretical methylation plot}
					\label{fig:PlotMeth}
				\end{subfigure}}
				\caption{A: Cytosine methylation status is shown for six stages of B cell development. Concentric circles are labelled with appropriate stage. B: Theoretical presentation of methylome data. A map would be produced for each of the 10 B cell subpopulations. C: Theoretical presentation of median methylation data. Each column represents the developmentally similar group of samples labelled.}
			\end{figure}
			
			
			
			It is expected that in both THI and control samples in all age groups, global methylation decreases as the B cells mature, as previously observed \citep{Kulis15,Lai13,Kulis12,Shaknovich11}. 
			The ChIP Seq data will allow us to produce a heatmap such as that in figure \ref{fig:TFBSheatmap} which displays the correlations between transcription factor binding sites and differentially methylated regions. 
			In the THI samples, we expect to find that regions related to B cell maturation will be hypermethylated compared to background levels. 
			An example of a possible region is Pax5. {\Huge Finish section here}
			In this cool paper, methylation of Pax5 was found to be cause arrest of B cell maturation at pro-B cell stage. 
			
			
			
			\begin{figure}[tb]
				\centering
				\includegraphics[width=.8\linewidth]{../Figures/TFbsKulis}
				\caption{Association of differentially methylated regions (DMRs) and transcription factor binding sites (TFBSs). From \citet{Kulis15}.}
				\label{fig:TFBSheatmap}
			\end{figure}
		
	\section{Aim 3:}
	
		\subsection{Hypothesis}
			
			Samples from THI patients will show decreased expression for B cell commitment associated genes compared to controls.
					
		\subsection{Proposed experiment}
			
			To investigate expression changes across the genome of THI samples, we will conduct RNA-Seq analysis on each of the B cell subsets in figure \ref{fig:BCellSorting}. 

			RNA-Seq libraries will be generated using the TruSeq Stranded Total RNA kit (Illumina). 
			Sequenced reads will be aligned to the genome using the STAR algorithm (v2.4.2a)\citep{Dobin13} and RPKM values will be calculated for each gene.
			qPCR analysis of gene expression will then be undertaken. 
			RNA will be reverse transcribed into cDNA and analysed using the Universal Probe Library System. 
			Target gene expression will be presented relative to average expression for the housekeeping genes \textit{GAPDH}, \textit{ACTB} and \textit{HPRT1}.
			
		
		\subsection{Possible outcomes and interpretations}				
			
			A graph such as that produced by \citet{Tallmadge15} (figure \ref{fig:TallTFexp}) will be produced to show the difference in expression of each implicated TF in each stage of B cell development.
			
				
			
			\begin{figure}
				\centering
				\includegraphics[width=\linewidth]{../Figures/TallmadgeTFexp}
				\caption{Expression of B lymphocyte-specific genes during development from hematopoietic stem cells to plasma cells. From \citet{Tallmadge15}. Gene expression presented in counts per million reads (CPM)}
				\label{fig:TallTFexp}
			\end{figure}
					
%\appendix
	\section{Appendices}
		
		\subsection{Sample information}

			
			
			
			
			\begin{landscape} % Sample table
				\footnotesize
				\begin{longtable}[c]{|l | l |p{4cm}|l|l|l|l|l|}
					\caption{Clinical details of patients with THI. Abbreviations: m, months; y, years; CVI, common variable immunodeficiency; -, unknown; def, d.} \\ \hline 	
					Individual & Tissue & Age at test  & Gender  & Ethnicity   & Breastfeeding status & Mode of birth & Family history of PID    \\ \hline \hline
					\endfirsthead
					\hline
					Individual & Tissue & Age at test  & Gender  & Ethnicity   & Breastfeeding status & Mode of birth & Family history of PID    \\ \hline \hline
					\endhead
					\hline 	\endfoot 
					\label{table:samples}
					1  & Whole Blood & 7m, 11m, 1y 7m, 2y 1m                                    & Female & Caucasian & Breastfed  & Vaginal           & None                 \\
					   & Tonsils     & 2y 1m                                                    &        &           &            &                   &  \\ \hline
					2  & Whole Blood & 8m, 11m, 1y 3m, 1y 6m, 1y 8m, 2y 1m, 2y 4m, 2y 6m, 2y 8m & Male   & Caucasian & Breastfed  & Vaginal           & None                 \\
					   & Tonsils     & 2y 6m                                                    &        &           &            &                   &  \\ \hline
					3  & Whole Blood & 9m, 1y 3m, 1y 9m, 2y 6m, 2y 9m, 3y 2m, 3y 6m             & Male   & Jewish    & Bottle fed & Caesarian section & None                 \\
					   & Tonsils     & 3y 2m                                                    &        &           &            &                   &  \\ \hline
					4  & Whole Blood & 10m, 1y 1m, 1y 8m                                        & Male   & Caucasian & Bottle fed & Vaginal           & None                 \\ \hline
					5  & Whole Blood & 9m, 11m, 1y 3m, 1y 5m, 1y 8m                             & Male   & Asian     & Breastfed  & Vaginal           & Brother with CVI     \\ \hline
					6  & Whole Blood & 7m, 9m, 11m, 1y 8m                                       & Female & Jewish    & Breastfed  & Vaginal           & None                 \\ \hline
					7  & Whole Blood & 9m, 11m, 1y 2m, 1y 6m                                    & Male   & -         & Breastfed  & -                 & -                    \\ \hline
					8  & Whole Blood & 7m, 9m,                                                  & Female & -         & Breastfed  & -                 & -                    \\ \hline
					9  & Whole Blood & 7m, 9m, 11m, 1y 3m 1y 8m                                 & Male   & -         & Breastfed  & -                 & -                    \\ \hline
					10 & Whole Blood & 7m, 9m, 11m, 1y 2m, 1y 6m                                & Male   & -         & Breastfed  & -                 & -                    \\ \hline
					11 & Whole Blood & 9m, 11m, 1y 1m, 1y 3m 1y 8m                              & Male   & -         & Bottle fed & -                 & -                    \\ \hline
					12 & Whole Blood & 7m, 11m, 1y 2m, 1y 11m, 2y 6m, 2y 9m, 3y 4m, 3y 7m, 4y   & Female & Jewish    & Breastfed  & Vaginal           & None                 \\
					   & Tonsils     & 2y 11m                                                   &        &           &            &                   &  \\ \hline
					13 & Whole Blood & 9m, 11m, 1y 2m, 1y 6m, 2y 4m, 2y 8m                      & Male   & Jewish    & Bottle fed & Vaginal           & None                 \\ \hline
					14 & Whole Blood & 9m, 1y 1m, 1y 7m, 2y 2m, 2y 9m, 3y 2m                    & Female & Caucasian & Bottle fed & Vaginal           & None                 \\
					   & Tonsils     & 3y 2m                                                    &        &           &            &                   &  \\ \hline
					15 & Whole Blood & 7m, 11m, 1y 1m                                           & Male   & Caucasian & Breastfed  &                   & Sister with IgA def. \\ \hline
					16 & Whole Blood & 8m, 1y 1m, 1y 4m, 1y 6m                                  & Male   & -         & -          & Vaginal           & -                    \\ \hline
					17 & Whole Blood & 9m, 1y 2m, 1y 6m                                         & Male   & -         & -          & Vaginal           & -                    \\ \hline
					18 & Whole Blood & 7m, 9m, 1y 1m                                            & Female & -         & -          & Caesarian section & -                    \\ \hline
					19 & Whole Blood & 7m, 9m, 11m, 1y 2m, 1y 6m                                & Male   & -         & -          & Vaginal           & -                    \\ \hline
					20 & Whole Blood & 9m, 11m, 1y 2m, 1y 6m                                    & Male   & -         & -          & Vaginal           & -                    \\ \hline
					21 & Whole Blood & 11m, 1y 3m, 1y 5m, 1y 8m                                 & Male   & -         & -          & Caesarian section & -                    \\ \hline
					22 & Whole Blood & 7m, 9m, 11m, 1y 8m                                       & Female & -         & -          & -                 & None                 \\ \hline
					23 & Whole Blood & 9m, 1y 2m, 1y 6m                                         & Male   & -         & -          & -                 & None                 \\ \hline
					24 & Whole Blood & 7m, 9m, 1y 1m, 1y 6m, 1y 11m, 2y 1m                      & Female & -         & -          & -                 & None                 \\
					   & Tonsils     & 2y 4m                                                    &        &           &            &                   &  \\ \hline
					25 & Whole Blood & 7m, 11m, 1y 3m, 1y 5m, 1y 8m                             & Male   & -         & -          & -                 & Sister with IgA def. \\ \hline
					26 & Whole Blood & 8m, 1y 1m,                                               & Male   & -         & -          & -                 & None
				\end{longtable}
			\end{landscape}	
				
	\bibliographystyle{BrittSuperScript}
	\bibliography{../literature}
	
%	\includepdf[pages=-]{../Reading/Transient_hypogammaglobulinemia_of_infancy_Ovadia_and_Dalal_2014.pdf}
%	\includepdf[pages=-]{../Reading/Whole-genome-fingerprint-of-the–DNA-methylome-during-human-B-cell-differentiation.pdf}
	
	
	
\end{document}