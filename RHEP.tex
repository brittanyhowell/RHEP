\documentclass[12pt]{article}
\usepackage[margin = 2.4cm]{geometry} % For margins of 3cm
\usepackage{gensymb} % For some symbols
\usepackage{amsfonts, amssymb, amsmath} % All three for maths symbols
\usepackage[export]{adjustbox} % For figure frames
\setlength{\parskip}{6pt} % To make nice looking paragraph spacing
\usepackage[export]{adjustbox} % For figure frames
\usepackage[numbers, super, sort&compress]{natbib} % bibliographies
\usepackage{setspace} % Allows double spacing
\usepackage{pdfpages} % Allows including PDFs
\usepackage{longtable} % Allows longtables
\usepackage{lscape} % To allow some pages to be in landscape
\usepackage{float, subfloat} % For H flag and subfloats
\usepackage{caption, subcaption} % for ABCD Subcaptions
% Longtable settings
\setlength\LTleft{0pt} % flush left
\setlength\LTright{0pt} % flush right
\setlength{\LTcapwidth}{8in}


\doublespacing % Makes all lines (not figure legends) double spaced

\title{Methylation status of B cells in Transient Hypogammaglobulinemia of Infancy}
\date{}
\author{Brittany Howell}

\newcommand{\CDH}{CD4$^+$ } %Thelper
\newcommand{\CDR}{CD4$^+$CD25$^+$ } %Treg
\newcommand{\IFNG}{IFN-$\gamma$ }
\newcommand{\NFKB}{NF-$\kappa$B }
\newcommand{\naive}{na\"{i}ve }

\newcommand{\Th}{T$_\textrm{H}$ }
\newcommand{\ThO}{T$_\textrm{H}1$} 
\newcommand{\ThT}{T$_\textrm{H}2$ }
\newcommand{\Tc}{T$_\textrm{C}$ }

%%%%%%%%%%%%%%%%%%%%%%%%%%%%%%%%%%%%%%%
%%%%%%%%%  Word count: 2500  %%%%%%%%%%  
%%%%%  Inc. all legends & titles  %%%%%  
%%%%%%%%%%%%%%%%%%%%%%%%%%%%%%%%%%%%%%%


\begin{document}
	\maketitle
	
	\paragraph{Hypothesis}
	~\\
	B cells from Transient Hypogammaglobulinemia of Infancy (THI) patients exhibit more DNA methylation, indicating incomplete B cell differentiation, than individuals without THI.
	
	\section{Background}
	
		Immunoglobulins are a vital component of the adaptive immune system \citep{Simon15}.  
		The B cells which produce antibodies (active immunoglobulin) in adults are not fully mature in young infants, hence serum immunoglobulin levels decrease after birth \citep{Martin10,Rechavi15,Stiemh80}. 
		Physiologic hypogammaglobulinemia refers to the point when serum immunoglobulin reaches its lowest level, commonly at 4 to 6 months of age \citep{Dressler89}. 
		THI is a disorder where regular hypogammaglobulinemia is prolonged or exacerbated then spontaneously alleviated. \citep{Stiemh80,Dressler89,AlHerz14,Gitlin56,AlHerz11,Rosen66,McGeady87, Dalal98,Tiller78,Buckley83}.
		The mechanism causing low serum immunoglobulin in THI patients is unknown \citep{AlHerz14}. 

	%%%%%%%%%%%%
	%%% Add paragraph about B cell development, so the stages we pick have context.
	%%%%%%%%%%%%
		\paragraph{Cause of THI} 
			~\\
			Immunoglobulin deficiency can result from B cell precursors failing to develop into mature B cells or mature B cells failing to differentiate into antibody secreting plasma cells \citep{Fiorilli86}. 
			Studies investigating THI have found that levels of circulating B cells are normal and subpopulations of B cells are intact \citep{Tiller78,Stiemh80,Siegel81,Buckley83,Fiorilli86,Dressler89}.
			With no obvious B cell deficiency, the cause of THI has been speculated extensively, but no proposed cause has been supported by replicated evidence \citep{Fudenberg64,Rosen66,Nathenson71,Willenbockel60,Soothill68,Tiller78,Fiorilli86,Ovadia14,Siegel81,McGeady87}.
			
			In regards to genetic inheritance, THI was initially thought to be familial \citep{Willenbockel60}. 
			\citet{Soothill68} proposed that THI was a manifestation of genetic heterozygosity for other immunodeficiency diseases, noting the high number of patients who had immunodeficient relatives.
			While it remains a possibility, no proceeding studies have shown supporting evidence \citep{Tiller78,Fiorilli86, Ovadia14}.
			
			
		\subsection{Lineage commitment}
	
			Activation and differentiation of T cells is governed greatly by epigenetic changes which affirm the lineage \citep{Zeng13}.
			DNA methylation was both the first epigenetic mechanism recognised, and the one most extensively studied \citep{Begin14}. 
			In T regulatory cells (Treg), the methylation status of the Treg-specific demethylated region (TSDR) is imperative in Treg differentiation \citep{Polansky08}.
			Tregs are induced by T cell receptor engagement which causes demethylation at the TSDR.
			FOXP3, the Treg inducing protein, can then bind to its own gene stabilising its expression and hence stabilising differentiation to Treg lineage.
			FOXP3 is also expressed during the activation of other T cell subsets, but due to the methylation of the TSDR, FOXP3 expression is transient \citep{Ohkura13}.
			Therefore, demethylation permits FOXP3 binding and thence confirms Treg lineage.
	
%	\section{Proposal}
%
%		
%			The most intriguing feature of THI is its self-limited nature: recurrent infections gradually subside and serum IgG levels increase with no obvious cause  \citep{Tiller78,Soothill68,Siegel81,McGeady87,Dressler89,Kowalczyk97,Dalal98}. 
%			Furthermore, the lack of evidence supporting a genetic basis suggests that the cause of THI is not within the genome \citep{Tiller78,Fiorilli86,Ovadia14}.
%			
%		
%			In common variable immunodeficiency, a disease related to THI, some B cells resemble immature B cells producing very little IgG \citep{Fiorilli86}. 
%			Incomplete maturation results in limited IgG production, so it is possible that the delayed onset of IgG synthesis in THI is also due to incomplete B cell maturation.
%			Development and differentiation are greatly influenced by epigenetic changes; latent maturity could be caused by inappropriate methylation of B cell development or differentiation genes. 
%	
%			To study incomplete lineage commitment B cells will be sampled from THI patients and age-matched controls then characterised using whole-genome bisulfite sequencing.
%			As in \citet{Kulis15} DNA methylation maps will be generated for sorted human B cell populations: uncommitted haematopoietic progenitor cells, pre-BII cells, plasma cells from bone marrow, germinal center B cells, \naive B cells from peripheral blood and memory B cells from peripheral blood.
%			Global demethylation normally occurs as B cells mature \citep{Oakes16}. 
%			If methylation is a cause of delayed maturation, the methylome of THI patients should be distinct to the age-matched controls.
			
	\section{Experimental aims}
		
		\paragraph{Experimental aim 1:} Using FACS, describe the B cell subpopulations in THI patients and normal individuals throughout early development.
		
		
		\paragraph{Experimental aim 2:} Using whole genome bisulphite sequencing and ChIP Seq, identify regions of the genome which are differentially methylated in THI samples and controls.
		
		\paragraph{Experimental aim 3:} Using RNA-Seq, identify differentially expressed regions in THI samples.

	\section{Data collection and management}
	
		\subsection{Collecting samples}
			
			% Add topic sentence
			Peripheral blood B cells will be obtained from the buffy coat (figure \ref{fig:BuffyCoat}) of the THI samples, and controls. 
			Plasma cells, germinal centre B cells and \naive B cells will be isolated from processed tonsil samples
			Members from the Pediatric Department at Wolfson Medical centre have kindly agreed to provide whole blood and tonsil samples they have collected from 26 THI patients (see table \ref{table:samples}).
			Further peripheral blood and tonsil samples will be obtained from the Women's and Children's Hospital Immunodeficiency clinic. 
			Blood will be taken from THI patients between initial presentation and normalisation of IgG levels.
			Control samples will come from donations by healthy subjects. 
			A full clinical record will be kept for every control and experimental sample collected to allow the best possible matching of THI patients with healthy controls. 
		
			\subsubsection{Matching controls} 
			
				To determine B cell maturation changes caused by THI, it is imperative to compare samples that are at the closest possible environmental stage.
				Maturation of B cells integrates numerous internal and environmental signals, so controlling for developmental stage has significant challenges. 
				Where possible, THI samples will be matched to control samples. 
				The most important criteria to match will be age, as it is the most prominent indicator of immune system development. 
				Secondly, the method of both birth (vaginal or caesarian) and feeding (bottle or breastfed) will be considered, as both have large influence over the immune system \citep{Jakobsson14,Cho13,Brandtzaeg03,Rogier14,Gomez14}.
				Further factors such as gender and ethnicity will be matched if possible. 
				Finally the diet of patients will be standardised to reduce the environmental effect on B cell maturation.
				In the best scenario, all of the above criteria will be matched between control and experimental samples. 
				However, there are limitations in the scope of the matching. 
				In the samples provided, there are already gaps in the clinical history of samples (table \ref{table:samples}). 
				Without information such as ethnicity or mode of birth, it is impossible to match a sample to an appropriate control. 
				Additionally, the control samples will be taken from participants' donations which may not match the clinical history of the THI samples.
				Even if it is possible to match samples according to stringent criteria, there are a myriad more external factors that will affect the dynamic nature of the immune system.
				 
				
		
		\subsection{Segregation of B cells}
			
			The maturation of B cells is a continuous process. 
			It is therefore important to ensure that changes which are observed between experimental and control samples are due to true THI effects, not normal B cell maturation changes.
			To reduce the variation between compared cells, the B cells will be segregated by developmental stage. 
			Preparation and segregation of B cells will be undertaken using FACS as in \citet{Kulis15} and \citet{Oakes16}.
			Subpopulations obtained will include \naive B cells from peripheral blood and tonsils, germinal center B cells and plasma cells from tonsils, and memory B cells from peripheral blood samples (figure \ref{fig:BCellSorting}). 
			There are always limitations when applying discrete developmental stages to continuous processes.
			However, segregating B cells into such specific subtypes significantly reduces the chance of detecting a differences which are due to developmental stage. 
			
			
			
				\begin{figure}[tb]
					\centering
					\begin{subfigure}[b]{.7\textwidth}
						\flushleft
						\includegraphics[width=\textwidth]{../Figures/BcellSorting4}
						\includegraphics[width=.78\textwidth]{../Figures/BcellSorting3}
						\caption{B cell populations}
						\label{fig:BCellSorting}
					\end{subfigure}
					\begin{subfigure}[b]{0.29\textwidth}
						\centering
						\includegraphics[width=\textwidth]{../Figures/BuffyCoat}
						\caption{Buffy coat}
						\label{fig:BuffyCoat}
					\end{subfigure}
					\caption{\ref{fig:BCellSorting}: Description of the FACS sorting markers used to segregate B cell populations. \ref{fig:BuffyCoat}: The buffy coat is the fraction of an anticoagulated blood sample which contains leukocytes and platelets.}
				\end{figure}`
	
	\section{Aim 1:}
		
		\paragraph{Hypothesis:} DNA of B cells from THI patients will have differentially methylated regions compared to normal individuals.
		
		\subsection{Proposed experiment}
			
			Whole genome bisulfite sequencing (WGBS) should be performed on two sets of biological replicates for seven B cell subpopulations specified in figure \ref{fig:BCellSorting}. 
			Two rounds of bisulfite conversion should be performed to ensure a conversion rate of over 99\%.
			2 x 100bp paired-end DNA sequencing will be undertaken using an Illumina HiSeq 2000 platform. 
			Read mapping will be carried out using the STAR aligner (v2.4.2a) \citep{Dobin13}.
			
			From the methylation data, median CpG methylation can be compared between THI and control samples at all seven stages of B cell development. 
			
		
		\subsection{Possible outcomes and interpretations}		
			
			A global decrease in methylation is expected to be observed, as previously reported
			 \citep{Kulis15,Lai13,Kulis12,Shaknovich11} 
			\begin{itemize}
				\item Methylome maps will be produced for every sample. 
				\item Hence, we will be able to compare across two vectors: normal vs THI and normal as time progresses
				\item I plan to produce a normal vs THI map for each age, and see the differences ebb as the IgG levels normalise
			\end{itemize}
	
	\section{Aim 2:}
	
		\paragraph{Hypothesis}
		
		\subsection{Proposed experiment}
		
		\subsection{Possible outcomes and interpretations}		
		
	\section{Aim 3:}
	
		\subsection{Hypothesis}
		
		\subsection{Proposed experiment}
		
		\subsection{Possible outcomes and interpretations}				
	

\appendix
	\section{Appendices}
		
		\subsection{Sample information}

			
			
			
			
			\begin{landscape} % Sample table
				\footnotesize
				\begin{longtable}[c]{|l | l |p{4cm}|l|l|l|l|l|}
					\caption{Clinical details of patients with THI. Abbreviations: m, months; y, years; CVI, common variable immunodeficiency; -, unknown; def, d.} \\ \hline 	
					Individual & Tissue & Age at test  & Gender  & Ethnicity   & Breastfeeding status & Mode of birth & Family history of PID    \\ \hline \hline
					\endfirsthead
					\hline
					Individual & Tissue & Age at test  & Gender  & Ethnicity   & Breastfeeding status & Mode of birth & Family history of PID    \\ \hline \hline
					\endhead
					\hline 	\endfoot 
					\label{table:samples}
					1  & Whole Blood & 7m, 11m, 1y 7m, 2y 1m                                    & Female & Caucasian & Breastfed  & Vaginal           & None                 \\
					   & Tonsils     & 2y 1m                                                    &        &           &            &                   &  \\ \hline
					2  & Whole Blood & 8m, 11m, 1y 3m, 1y 6m, 1y 8m, 2y 1m, 2y 4m, 2y 6m, 2y 8m & Male   & Caucasian & Breastfed  & Vaginal           & None                 \\
					   & Tonsils     & 2y 6m                                                    &        &           &            &                   &  \\ \hline
					3  & Whole Blood & 9m, 1y 3m, 1y 9m, 2y 6m, 2y 9m, 3y 2m, 3y 6m             & Male   & Jewish    & Bottle fed & Caesarian section & None                 \\
					   & Tonsils     & 3y 2m                                                    &        &           &            &                   &  \\ \hline
					4  & Whole Blood & 10m, 1y 1m, 1y 8m                                        & Male   & Caucasian & Bottle fed & Vaginal           & None                 \\ \hline
					5  & Whole Blood & 9m, 11m, 1y 3m, 1y 5m, 1y 8m                             & Male   & Asian     & Breastfed  & Vaginal           & Brother with CVI     \\ \hline
					6  & Whole Blood & 7m, 9m, 11m, 1y 8m                                       & Female & Jewish    & Breastfed  & Vaginal           & None                 \\ \hline
					7  & Whole Blood & 9m, 11m, 1y 2m, 1y 6m                                    & Male   & -         & Breastfed  & -                 & -                    \\ \hline
					8  & Whole Blood & 7m, 9m,                                                  & Female & -         & Breastfed  & -                 & -                    \\ \hline
					9  & Whole Blood & 7m, 9m, 11m, 1y 3m 1y 8m                                 & Male   & -         & Breastfed  & -                 & -                    \\ \hline
					10 & Whole Blood & 7m, 9m, 11m, 1y 2m, 1y 6m                                & Male   & -         & Breastfed  & -                 & -                    \\ \hline
					11 & Whole Blood & 9m, 11m, 1y 1m, 1y 3m 1y 8m                              & Male   & -         & Bottle fed & -                 & -                    \\ \hline
					12 & Whole Blood & 7m, 11m, 1y 2m, 1y 11m, 2y 6m, 2y 9m, 3y 4m, 3y 7m, 4y   & Female & Jewish    & Breastfed  & Vaginal           & None                 \\
					   & Tonsils     & 2y 11m                                                   &        &           &            &                   &  \\ \hline
					13 & Whole Blood & 9m, 11m, 1y 2m, 1y 6m, 2y 4m, 2y 8m                      & Male   & Jewish    & Bottle fed & Vaginal           & None                 \\ \hline
					14 & Whole Blood & 9m, 1y 1m, 1y 7m, 2y 2m, 2y 9m, 3y 2m                    & Female & Caucasian & Bottle fed & Vaginal           & None                 \\
					   & Tonsils     & 3y 2m                                                    &        &           &            &                   &  \\ \hline
					15 & Whole Blood & 7m, 11m, 1y 1m                                           & Male   & Caucasian & Breastfed  &                   & Sister with IgA def. \\ \hline
					16 & Whole Blood & 8m, 1y 1m, 1y 4m, 1y 6m                                  & Male   & -         & -          & Vaginal           & -                    \\ \hline
					17 & Whole Blood & 9m, 1y 2m, 1y 6m                                         & Male   & -         & -          & Vaginal           & -                    \\ \hline
					18 & Whole Blood & 7m, 9m, 1y 1m                                            & Female & -         & -          & Caesarian section & -                    \\ \hline
					19 & Whole Blood & 7m, 9m, 11m, 1y 2m, 1y 6m                                & Male   & -         & -          & Vaginal           & -                    \\ \hline
					20 & Whole Blood & 9m, 11m, 1y 2m, 1y 6m                                    & Male   & -         & -          & Vaginal           & -                    \\ \hline
					21 & Whole Blood & 11m, 1y 3m, 1y 5m, 1y 8m                                 & Male   & -         & -          & Caesarian section & -                    \\ \hline
					22 & Whole Blood & 7m, 9m, 11m, 1y 8m                                       & Female & -         & -          & -                 & None                 \\ \hline
					23 & Whole Blood & 9m, 1y 2m, 1y 6m                                         & Male   & -         & -          & -                 & None                 \\ \hline
					24 & Whole Blood & 7m, 9m, 1y 1m, 1y 6m, 1y 11m, 2y 1m                      & Female & -         & -          & -                 & None                 \\
					   & Tonsils     & 2y 4m                                                    &        &           &            &                   &  \\ \hline
					25 & Whole Blood & 7m, 11m, 1y 3m, 1y 5m, 1y 8m                             & Male   & -         & -          & -                 & Sister with IgA def. \\ \hline
					26 & Whole Blood & 8m, 1y 1m,                                               & Male   & -         & -          & -                 & None
				\end{longtable}
			\end{landscape}	
				
	\bibliographystyle{BrittSuperScript}
	\bibliography{../literature}
	
	\includepdf[pages=-]{../Reading/Transient_hypogammaglobulinemia_of_infancy_Ovadia_and_Dalal_2014.pdf}
	\includepdf[pages=-]{../Reading/Whole-genome-fingerprint-of-the–DNA-methylome-during-human-B-cell-differentiation.pdf}
	
	
	
\end{document}