\documentclass[12pt]{article}
\usepackage[margin = 2.4cm]{geometry} % For margins of 3cm
\usepackage{float} % For H float position
\usepackage{gensymb} % For some symbols
\usepackage{amsfonts, amssymb, amsmath} % All three for maths symbols
\usepackage[export]{adjustbox} % For figure frames
\setlength{\parskip}{6pt} % To make nice looking paragraph spacing
\usepackage[export]{adjustbox} % For figure frames
\usepackage[numbers, super, sort&compress]{natbib} % bibliographies
\usepackage{setspace} % Allows double spacing

\doublespacing % Makes all lines (not figure legends) double spaced

\title{Methylation status of B cells of those afflicted with Transient Hypogammaglobulinemia}
\date{Research Hypothesis and Experimental Proposal}
\author{Brittany Howell \\ a1646948}

\newcommand{\CDH}{CD4$^+$ } %Thelper
\newcommand{\CDR}{CD4$^+$CD25$^+$ } %Treg
\newcommand{\IFNG}{IFN-$\gamma$ }
\newcommand{\NFKB}{NF-$\kappa$B }
\newcommand{\naive}{na\"{i}ve }

\newcommand{\Th}{T$_\textrm{H}$ }
\newcommand{\ThO}{T$_\textrm{H}1$} 
\newcommand{\ThT}{T$_\textrm{H}2$ }
\newcommand{\Tc}{T$_\textrm{C}$ }

\begin{document}
	\maketitle
	%	\tableofcontents
	
	\section{Background}
	
		Immunoglobulins are a vital component of the adaptive immune system \citep{Simon15}.  
		The B cells which produce antibodies (active immunoglobulin) in adults are not fully mature in young infants, resulting in a decrease in serum immunoglobulin levels after birth \citep{Martin10,Rechavi15,Stiemh80}. 
		Physiologic hypogammaglobulinemia refers to the point when serum immunoglobulin reaches its lowest point, commonly at 4-6 months of age \citep{Dressler89}. 
		Transient hypogammaglobulinemia (THI) is a disorder whereby affected persons have a prolongation or exacerbation of regular hypogammaglobulinemia, followed by spontaneous recovery \citep{Stiemh80,Dressler89,AlHerz14,Gitlin56,AlHerz11,Rosen66,McGeady87, Dalal98,Tiller78,Buckley83}.
		The mechanism causing low serum immunoglobulin in THI patients has not yet been elucidated \citep{AlHerz14}. 

	
		\paragraph{Proposed mechanism} 
			~\\
			Immunoglobulin deficiency can result from B cell precursors failing to either mature into B cells or further fail to differentiate into antibody secreting plasma cells \citep{Fiorilli86}. 
			Studies investigating THI have found that levels of circulating B cells are normal and subpopulations of B cells are intact \citep{Tiller78,Stiemh80,Siegel81,Buckley83,Fiorilli86,Dressler89}.
			With no obvious B cell deficiency, the cause of THI remains unknown \citep{AlHerz14}. 
			
			\citet{Fudenberg64} first proposed that the fetal IgG molecules produced during pregnancy were enough to stimulate an immune response from the mother. 
			The immune response was thought to produce maternal antibodies which caused transient suppression of fetal immunoglobulin production. 
			\citet{Rosen66} and \citet{Nathenson71} were unable to find agglutinators in four mothers of infants with THI.
			\citet{Willenbockel60} observed THI as familial, \citet{Soothill68} further proposed THI as a manifestation of genetic heterozygosity for other immunodeficiency diseases, noting the high number of patients who had immunodeficient relatives.
			While it remains a possibility as noted by \citet{McGeady87}, no studies have shown supporting evidence \citep{Tiller78,Fiorilli86, Ovadia14}.
			\citet{Siegel81} hypothesised that a T cell deficiency was the cause after the observation of low T cell numbers in THI patients, but T cell numbers have been normal in proceeding studies.
			%
			% Could easily delete
			%
			\citet{McGeady87} suggested that frequent antibiotic treatment could induce hypogammaglobulinemia by diminishing bacteria gut flora. 
			The explanation was not described in any other THI literature, and was indeed described by \citet{McGeady87} to be improbable due to the mostly brief courses of antibiotic treatments given to THI patients.
			In summary, a wealth of theories have been proposed for mechanism of THI but none have been supported by replicated evidence.
			
	\section{Lineage commitment}

		Activation and differentiation of T cells is governed greatly by epigenetic changes which insure the phenotype of the T cell \citep{Zeng13}.
		DNA methylation was the first epigenetic mechanism recognised, and the one that is most extensively studied \citep{Begin14}. 
		In T regulatory cells (Treg), the methylation status of the Treg-specific demethylated region (TSDR) is imperative in Treg differentiation \citep{Polansky08}.
		In the thymus, where T cells mature, Tregs are induced by T cell receptor engagement. 
		Subsequent demethylation occurs at the TSDR allowing FOXP3 to bind to its own gene to stabilise FOXP3 expression, stabilising differentiation to Treg.
		FOXP3 is also expressed during the activation of other T cell subsets, but due to the methylation of the TSDR, FOXP3 expression is transient \citep{Ohkura13}.
		Therefore, demethylation permits FOXP3 binding and thence confirms Treg lineage.
		
	\section{Proposal}
	
		The most intriguing feature of THI is its self-limited nature; that gradually recurrent infections subside and serum IgG levels increase with no obvious cause  \citep{Tiller78,Soothill68,Siegel81,McGeady87,Dressler89,Kowalczyk97,Dalal98}. 
		An investigation into common variable immunodeficiency, a disease related to THI, B cells resembling immature B cells were observed \citep{Fiorilli86}. 
		The observed cells were found to produce little IgG. 
		It is possible that the delayed onset of IgG synthesis in THI is due to incomplete lineage commitment, caused by inappropriate methylation of differentiation genes. 
		In normal B cell maturation, global methylation loss is observed \cite{Oakes16}.
		\citet{Tagoh04} explained that epigenetic programs are engraved into the chromatin of lineage-specific genes before cell lineage specification and the onset of detectable gene expression.
		
		To investigate incomplete lineage commitment, B cells will be sampled from THI patients and age-matched controls and tested for methylation of key development and differentiation genes. 
		Transcription factor families which show hypomethylation in B cell development include are AP-1, EBF, RUNX, OCT, IFF and NF$\kappa$B
		Genes crucial for B cell identity (\textit{Pax5, Spib, Ebf1}) maintain an active epigenetic state \citep{Li13,Choukrallah14}. 
		Using bisulfite sequencing, the methylation status of loci imperative for B cell development can be established. 
		
		\paragraph{Hypothesis}
			~\\
			At key B cell development loci, THI patients will exhibit more DNA methylation than age matched controls.


	
	
	\newpage
	%	\bibliographystyle{brittany-superscript}
	\bibliographystyle{BrittSuperScript}
	\bibliography{../literature}
	
\end{document}